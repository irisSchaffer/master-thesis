\chapter{Interactive Mechanisms}
\label{cha:mechanisms}

In a first step of identifying possible mechanisms which could make presentations more engaging and interactive, we analysed different types of presentations. There are several factors which determine these types, such as the size of the audience, the environment around and purpose of the presentation as well as the speaker and audience. In the following these factors will be described shortly to then present and discuss mechanisms each of them could profit from most.

\section{Factors}

\subsection{Audience Size}
One aspect which plays an important role in the type of presentation and thereby interactive mechanisms applicable is the size of the audience. Different challenges present themselves, depending on the amount of listeners: While there might be a dialogue between speaker and audience in small group sizes, it is hard for audience members to directly communicate with a speaker during conferences or in large lecture halls. Shy or introvert attendees might remain unheard \cite{Bry:Backstage} and only a usually randomly chosen subset of people get the opportunity to ask audience questions after talks in conferences. At the same time estimating the audience's knowledge and interest as well as the general mood gets increasingly difficult both for the presenter and attendee as the number of participants rises. 
The general conclusion therefore is that big audiences struggle to connect and interact with the speaker and each other and interactive tools must aim to strengthen the bidirectional bond between presenter and listeners. In smaller groups, on the other hand, the focus should be put on supporting the already existing dialogue and exchange between all participants of the presentation (speaker and audience members). As peer-pressure might rise the more listeners know each other, ways of anonymously contributing to the outcome or flow of a presentation might be appreciated.

\subsection{Presentation Environment}
% Remote vs co-located!
% setting: classroom vs. meeting vs. conference etc.
The environment of a presentation is described by all factors surrounding the presentation, e.g. the setting in which a talk is given, in other words, if it is embedded in a meeting, a talk at a conference or a lecture at school or university. Other aspects worth considering are whether the audience is co-located or distributed or which technologies are available. As this work concerns itself only with mobile devices in the context of co-located presentations, difficulties added through remote presentations as well as missing technical equipment will be disregarded in this section. Instead, a closer look will be taken at the setting: In a lecture, it is desirable to measure the students' participation and engagement, as well as their understanding of a topic. In meetings, on the other hand, interactive mechanisms are more likely to aim at promoting collaboration between all participants. Conferences might search to foster the interactivity between attendees, to support networking. Instead of taking all possible scenarios into consideration, this work concentrates on business-related settings and explores mechanisms which make it easier to collaborate on presentations.

\subsection{Presentation Purpose}
When speaking of the purpose of a presentation, McClain \cite{McClain:TypeOfPresentations} identifies four major types: informational, motivational, pursuasive and sales. According to him, informational presentations aim at educating the listeners, while motivational speeches try to inspire the audience to do something. Pursuasive talks usually present new ideas or directions and have the goal of making the listeners re-think old approaches and consider or even embrace new ones. Sales presentations, lastly, often use elements of the other three categories with the aim of ``obtaining a decision at the presentation's end'' (\cite{McClain:TypeOfPresentations}). While informational presentations 
% welches material wird mitgenommen? wie können lectures erinnert werden? welches material nehmen wir mit? --> nachbereitung/handouts!
% motivational und pursuasive: mehr im hier und jetzt, memorable auf gefühlsebene!
% in manchen ist es einfacher das Publikum leiten zu lassen als in anderen (motivational & pursuasive)

\subsection{Speaker and Audience}
% tech savvyness for audience
% spontanität and flexibility for speaker --> when do I take questions? überfordert? etc.
% level of formality!

\section{Resulting Mechanisms}
\chapter{Interactive Mechanisms}
\label{cha:mechanisms}

In a first step of identifying possible mechanisms which could make presentations more engaging and interactive, we analysed different types of presentations. There are several factors which determine these types, such as the size of the audience, the environment around and purpose of the presentation as well as the speaker and audience. In the following these factors will be described shortly to then present and discuss mechanisms these could profit from most.

\section{Factors}

\subsection{Audience Size}
One aspect which plays an important role in the type of presentation and thereby the interactive mechanisms applicable is the size of the audience. Different challenges present themselves, depending on the amount of listeners: While there might be a debate between speaker and audience in small group sizes, it is hard for audience members to directly communicate with a speaker during conferences or in large lecture halls. Shy or introvert attendees might remain unheard \cite{Bry:Backstage} and only a usually randomly chosen subset of people get the opportunity to ask audience questions after talks in conferences. At the same time estimating the audience's knowledge and interest as well as the general mood gets increasingly difficult both for the presenter and attendee as the number of participants rises. Additionally to the interaction between speaker and audience, another important factor is listener-listener interaction \cite{Moore:ThreeTypesOfInteraction}. Group-dynamics largly depend on the audience size and smaller groups usually perform better than big ones \cite{Phillips:GroupProblemSolving}.
The general conclusion therefore is that big audiences struggle to connect and interact with the speaker and each other and interactive tools must aim to strengthen the bidirectional bond between presenter and listeners. In smaller groups, on the other hand, the focus should be put on supporting the already existing dialogue and exchange between all participants of the presentation (speaker and audience members). As peer-pressure might rise in smaller groups and the better listeners know each other, ways of anonymously contributing to the outcome or flow of a presentation become more important.

\subsection{Presentation Environment}
% Remote vs co-located!
% setting: classroom vs. meeting vs. conference etc.
The environment of a presentation is described by all factors surrounding the presentation. One of them is the setting a talk is given in, in other words, if it is embedded in a meeting, a talk at a conference or a lecture at school or university. Other aspects worth considering are whether the audience is co-located or distributed and which technologies are available. As this work concerns itself only with mobile devices in the context of co-located presentations, difficulties added through remote presentations as well as missing technical equipment will be disregarded in this section. Instead, a closer look will be taken at the setting: In a lecture, it is desirable to measure the students' participation and engagement, as well as their understanding of a topic. In meetings, on the other hand, interactive mechanisms are more likely to aim for the promotion of collaboration between all participants. Conferences might search to foster the interactivity between attendees, to support networking. Instead of taking all possible scenarios into consideration, this work concentrates on business-related settings and explores mechanisms which foster collaboration.

\subsection{Presentation Purpose}
When speaking of the purpose of a presentation, McClain \cite{McClain:TypeOfPresentations} identifies four major types: informational, motivational, pursuasive and sales. According to him, informational presentations search to educate the listeners, while motivational speeches try to inspire the audience to take action. Pursuasive talks usually present new ideas or directions and have the goal of making the listeners re-think old approaches and consider or even embrace new ones. Sales presentations, lastly, often use elements of the other three categories with the aim of ``obtaining a decision at the presentation's end'' \cite{McClain:TypeOfPresentations}. While motivational, pursuasive and to some extend sales presentations often operate on an emotional level in the present moment, informational talks often include a way for listeners to re-visit the taught material through transcripts, lecture notes or handouts. Moreover, motivational, pursuasive and sales presentations focus on the goal of getting the audience to take action and therefore put more emphasise on the listeners than content-centric informational speeches. This creates two very distinctive needs for interactive mechanisms: on one side the ability for the audience to actively shape the path of the presentation, on the other hand the possibility to re-visit presentation slides (potentially including notes and additional material), after the end of a talk.

In the context of business-settings, as discussed in the present thesis, one should also take meeting-types into consideration. Böhmer et al. \cite{Bohmer:SmartphoneUseRude} differentiate between conversations, status updates, presentations (``scheduled session with higher level of formality'', not to be confused with the more general concept of presentations examined in this thesis), brainstorming and training. All these can be prepared beforehand and have different requirements and challenges to overcome: In brainstorming sessions, for example, it is vital for every team member to be able to contribute to the end result; in trainings, it is hardly possible to find a pace that fits all participants.
% welches material wird mitgenommen? wie können lectures erinnert werden? welches material nehmen wir mit? --> nachbereitung/handouts!
% motivational und pursuasive: mehr im hier und jetzt, memorable auf gefühlsebene!
% in manchen ist es einfacher das Publikum leiten zu lassen als in anderen (motivational & pursuasive)

\subsection{Speaker and Audience}
% tech savvyness for audience
% spontanität and flexibility for speaker --> when do I take questions? überfordert? etc.
% level of formality!
The last factor taken into consideration in this chapter are the speaker and listeners themselves. Depending on the individual interest, but also character traits such as introversion, listeners will be more or less likely to engage in a presentation actively \cite{Bry:Backstage}. The inter-attendee relationship as well as the relationship between attendees and speaker also plays a role in which mechanisms are appreciated and which are not \cite{Moore:ThreeTypesOfInteraction}: while it is common for listeners to jump into the role of the presenter in meetings with flat-hierarchies, the same behaviour is a rare sight in lectures or might even be deemed inappropriate and rude in more formal settings. When taking the speaker into consideration, the set of tools needed are more sophisticated than the ones necessary to merely follow a presentation: Foremostly, speakers need a way of navigating through slide decks. It is desirable to have an overview of the entire presentation and while listeners only concentrate on the current slide, many speakers rely on notes or use timers, which also need to be placed in the interface.
Moreover, the presenter's personal traits, experience and bluntly talent, play a central role in the successful deployment of interactive mechanisms: the flexibility, confidence and technological expertise of a presenter all determine how distracting or even stressful certain features are perceived as and whether a speaker is able to react to these spontaniously \cite{Wacker:PresenterExperience}. It is therefore crucial to give speakers the ability to turn said mechanisms on and off. An important challenge which also arises with this question is how to design these mechanisms in a way that is neither perceived as intrusive nor interrupting (this will be discussed in more detail in chapter \ref{cha:design}).
To summarise, when developing interaction tools, it is vital to take a participant's personality into consideration. In the context of presentation, shy listeners should be given tools to make them heard; presenters need full control of the mechanisms provided.

\section{Resulting Mechanisms}
With these aspects and challenges in mind, a multitude of different mechanisms can be derived. Although many more are thinkable, this section concentrates on the ones implemented in course of the thesis project. However, we will try to point out other possible features and provide resources to projects focusing on these.

% Speaker: Remote control
% Speaker: See next slide right and down, notes on phone or tablet/laptop!
%
% Following slides on personal device
% Questions
% Anonymity
% Polls
% Reactions (Emojis) -- tell how these can engage the people more, not only about how they provide feedback to presenter!
% Annotating slides and sharing content (!) --> not really studied yet, so cool!

\subsection{Remote Control}
One mechanism of special importance for speakers is the ability to control slides and navigate through them. As many presentations involve more than just one speaker and can profit from sharing control over slides with others \cite{Chattopadhyay:OfficeSocialRemoteControl}, any amount of presenters should be able to be connected at any given point. Controlling should be possible from any personal device, may it be a laptop, tablet or mobile phone, giving the speakers maximal freedom.

While using arrow-keys in a desktop environment feels natural to navigate between slides, the native equivalent on touch-devices are swipe gestures. These are more accurately and faster when operating a phone with one hand \cite{Lai:SingleHandedThumbInteraction} and are less prone to error when used eye-free \cite{Negulescu:TapSwipeMove} and should therefore be prefered over buttons, clustering the interface.

\subsection{Following Slides}
% also gives latecomers the chance...
For members of the audience, one important feature, both in the context of re-visiting slides, to accommodate individual learning paces and even to give latecomers a chance to catch up with the presentation, is to be able to independently follow the slides. This should again be possible on any personal device and focus on the slide content. The mechanism can be designed in many different ways and could even allow listeners to remote control the presentation \cite{Chattopadhyay:OfficeSocialRemoteControl}, our implementation however only provides individual slide navigation on the personal device. Additionally, the progress of the presentation should always be synchronised with the individual devices, allowing listeners to truely follow the slides. This basic mechanism can be extended to offer features such as turning the synchronisation on and off (effectively allowing to navigate freely and jump back to the current state) or to only allow listeners to see the last slide the presenter has already shown.

\subsection{Paths}

\subsection{Reactions}

\subsection{Asking Questions}

\subsection{Polls}

\subsection{Sharing Content}

\subsection{Speaker View}
Most of the mechanisms presented in this section involve both the speaker and audience. As the name already gives away, this is not the case for the speaker view. As described before, 
\chapter{Discussion}
\label{cha:discussion}

A system such as the one developed in course of this thesis can be evaluated in many ways: The performance of the software can be quantified by measuring response times of the application, as happened in \cite{Niwa:Web-presentation-powerpoint} or \cite{Inoue:RealTimeQuestionnaire}, the usability can be assessed quantitatively through usability tests, measuring error rates and time to complete certain tasks and qualitatively using \emph{loud thinking} and interviews \cite{Reindl:automatisierte-user-interface-evaluierung}. If the expectations lied out in chapter \ref{cha:mechanisms} were actually met, if listeners feel more engaged and whether the perception of phone usage in presentations has changed with unveil can also be estimated using qualitative and quantitative evaluation methods. We feel, however, that valid results can only be achieved observing the usage of the presented tool over a longer period of time and involving control groups. Since this would go beyond the scope of this master thesis, we decided to refrain from a formal study and instead rely on the informal observation and evaluation of the system, both during and after working on the project.
Throughout the process of developing the different libraries, several informal presentations and meetings have been held using unveil, both in academic and business settings. This allowed for several iterations of the design, as well as the gradual introduction of more and more interactive mechanisms. Our findings and the strengths and the shortcomings of the project will be discussed in this chapter, followed by an outlook on future work.

\section{Usability}
We were pleased to see that all listeners have so far understood the base interface of the application, both on their phones and on the laptop. We have however not had a single tablet user in any of the presentations yet and most of the mobile phones were iPhones running Chrome or Safari. Some users experienced glitches in the navigation of the slides in Safari and could scroll within them, moreover local storage is disabled in Safari's \emph{private mode} (thus making it impossible to persist voting results). On bright side, initial concerns regarding the usage of socket.io have not come true, since the presentations were always held in networks we had full control over (so corporate firewalls did not play a role).

As far as the response time of the application is concerned, with up to 30 concurrent users, no noticable declines in performance have been experienced so far and the real-time features feel instantaneous. The only current limitation to this is the transmission of base64-encoded media content through the WebSocket. Since the delay is between listener and speaker, however, the transmission time is not critical for the application to feel responsive. Another point connected to sharing media is the absence of a native sharing feature for phones, which was an anticipated limitation of creating a web application over a mobile one.

The sharing mechanisms were generally very well accepted by users and seem to have caused most excitement about the new technology. In our trials, however, we experienced a large amount of test-data being sent through the mechansisms, especially in the beginning of the presentation. We therefore advice others to provide one or two empty slides in the beginning of the presentation if the audience is new to the software, so the application and its functionality can be explored. Since some users mixed textual content with links in the early tests, we separated media upload, link sharing and questions (or other text) into distinct buttons. We were very pleased to see that some of the people who had used unveil in meetings before, later actively mentioned wanting to have the ability to share links and thoughts in meetings without the technology. Resulting from this feedback, an unexpected way of using the software emerged: Instead of enriching an existing presentation, an empty slide was provided, allowing anyone in a meeting to add their own content in the $2$-dimensional space in a brainstorming-like fashion.

The reaction-system was the one implemented last, which is why we are still missing more thorough insight into how listeners use it. In informal feedback rounds about the mechanism, the usage of emoji seemed to be understood well, however, these conversations were held with digital professionals and it will require further analysis to see the acceptence beyond other users. Another observation we made, is that although listeners generally reacted exceedingly positively to QR-codes (linking to the address of the presentation), it was usually faster for them to type the address into their browsers directly. This usually caused a few minutes of interruption, which should be accounted for.

The voting mechanism was also understood by users and no questions have yet arisen from it. To our surprise this functionality, like the possibility of having different paths through the presentation, did not seem to impress or excite users too much. This on one hand shows the implicitness of forms and voting mechanisms on the internet, on the other hand might indicate a potential for improvement.

The biggest weakness of unveil for most users was the inability for the presentation to be permanently altered. Especially in meetings and informational presentations, many listeners asked for a link to the collaboratively created presentation afterwards, some users were forced to reload the website due to cross-browser compatibility issues and lost the current state of the slides, late-comers also did not have the possibility to jump into an already altered presentation. This will make it necessary to create a more intelligent and opinionated back end in the future.

\section{Creation of Presentation}
Most of the feedback we have collected about unveil over the last months came from listeners. This has two reasons: Firstly, the creation of presentations requires knowledge and experience with front end web development technologies, secondly, we have not started promoting the resulting libraries yet, as we feel the system is not stable and mature enough to be used outside our internal settings. However, external developers have provided us with their feedback regarding the syntax used for defining slide decks and seem to not have had any problems understanding the usage of the libraries. This validates our decision to choose semantically-named tags rather than HTML class names to identify different components. On the downside, we seem to have overestimated the level of knowledge necessary to create own presentations, as a few developers were not entirely familiar with the new ES6 syntax and the process of bundling JavaScript and CSS files, so an easier way of importing all necessary libraries should be provided in an example presentation.
In the long run, this product will only be able to increase its popularity if a visual editor for authoring slides, as well as a system to manage (i.e. host) them will be available. Some users also raised the question of how and if it was possible to import PowerPoint presentations, to be able to use the created mechanisms in connection with already existing software.

\section{Architecture}
Generally, it has to be said that the project is still only a prototype and does not provide the stability necessary for us to feel confident promoting the product. One particular problem we have been experiencing seemingly randomly is the navigation getting stuck in an infinite redirect loop when two or more connected presenters try to navigate at the same time. 
Although the chosen architecture allows for a lot of flexibility and freedom for developers and in that sense fulfils the criteria impress.js and reveal.js did not, we have discovered more powerful, streamlined and widespread patterns when working with React over the last half year. Since we had no experience with React prior to the start of the development, best practices oftentimes only became apparent throughout the project and through the work on and with other React applications. To make the libraries yet easier to use and extend and more easily debuggable, future iterations of unveil will likely be based on \emph{Redux} \cite{redux}, with the presentation state globally accessible. This will make it easier to decouple components from each other through the introduction of a centralised state container.

\section{Future Work}
The biggest outstanding improvement is definitely the persistence of updates to the presentation state for later revision by both listeners and presenter. For the presenter to profit even more from the new possibilities of interactive presentations, we would like to provide some analytics: When did the audience interact with the presentation most? When exactly were which reactions triggered? Which slides provoked the most input or questions and how long did audience members stay on each slide when browsing through them individually?

As far as the interactive mechanisms implemented in this project are concerned, initial tests have shown potential for improvement as well: the question sharing tool was used very seldomly, but instead the wish for a commenting or annotating functionality was mentioned by some users. Votings could offer several types (e.g. multiple choice, open questions, ratings), as well as different ways of visualising the results (e.g. pie charts, stars, clusters of answers). They could moreover be more tightly linked to paths through the presentation, so the result of a voting could directly link to a path.
Reactions generally need more testing and although we are content with the current functionality for the scope of this thesis, analytics for the presenter and time-based rather than slide-based display of reactions are desireable.
Overall, more animations could make the presentation feel even more responsive and offer a more \emph{native} feel for mobile users \cite{GoogleMaterialDesignGuide}.

From an architectural point of view, it is our declared objective to deliver a stable first version of the unveil ecosystem within the coming months. Redux and even further separation of concerns will make it easier to automatically test the application and add new features, such as the persistence of presentation state to local storage or a database. Moreover, porting the existing system to React Native and adding native sharing possibilities is worth a consideration.

A possibility not at all touched in the present work is the development of an authoring and hosting tools for unveil presentations. This would make it easier to create slide decks and render the necessity for front end development knowledge obsolete; effectively giving anyone a tool to create interactive presentations. This would also be possible offering imports of PowerPoint presentations or even re-building the interactive mechanisms as PowerPoint plugins.

Another opportunity which arose from the users' desire to share to a blank presentation during meetings as well as questions about annotations, would be to follow a canvas-based rather than a slide-based approach in future projects. That way meeting participants could effectively use the platform as a tool of collaboratively creating and sharing content, to brainstorm and take notes from any device in a shared working space.

Finally, more formal observations and long-term studies will be necessary to quantify the success of the developed mechanisms. Currently, the tool is mostly in use for informal Monday morning presentations with other digital professionals; the acceptance and usability of the application will have to be re-visited and re-evaluated when assessed with less technologically versed users.
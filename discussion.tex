\chapter{Discussion}
\label{cha:discussion}

A system such as the one developed in course of this thesis can be evaluated in many ways: the performance of the software can be quantified by measuring response times of the application, as happened in \cite{Niwa:Web-presentation-powerpoint} or \cite{Inoue:RealTimeQuestionnaire}, the usability can be assessed qualitatively through usability tests, measuring error rates and time to complete certain tasks and qualitatively using \emph{loud thinking} and interviews \cite{Reindl:automatisierte-user-interface-evaluierung}. If listeners feel more engaged and whether the perception of phone usage in a presentation has changed can also be estimated using qualitative and quantitative evaluation methods, however, valid results can only be achieved observing the usage of the presented tool over a longer period of time and involving a control group. Since this would go beyond the scope of a master thesis, we instead decided to refrain from a formal study and instead rely on the informal observation and evaluation of the system, both during and after working on the project.
Throughout the process of developing the different libraries, several presentations and meetings have been held using the resulting libraries, both in informal and formal as well as academic and business settings. This allowed for several iterations of the design, as well as the gradual introduction of more and more interactive mechanisms. Our findings and the strengths and the shortcomings of the project will be discussed in this chapter, followed by an outlook on future work.

\section{Usability}
We were pleased to see that all listeners understood the base interface of the application, both on their phones and on the laptop. We have however not had a single tablet user in any of the presentations held so far and most of the mobile phones were iPhones running Chrome or Safari. Some users experienced glitches in the navigation of the slides and could scroll within them, moreover mobile Safari seemed to have difficulties with local storage (and thus persisting voting results), on some devices. On the other hand, the concerns regarding the usage of socket.io have not come true, since the presentations were always held in networks we had control over (so corporate firewalls did not play a role).

As far as the response time of the application is concerned, with up to 30 concurrent users, no noticable declines in performance have been experienced so far and the real-time features feel instantaneous. The only current limitation to this is the transmission of base64-encoded media content through the WebSocket. Since the delay is between listener and speaker, however, the transmission time is not critical for the application to feel responsive. Another point connected to this mechanism was that some users were missing the possibility to natively share content from their phones, which was an anticipated limitation of creating a web application over a mobile one.
The sharing features have generally been accepted by users and seemed to have caused most excitement about the new technology. In our trials, however, we experienced a large amount of test-data being sent through the mechansisms, especially in the beginning of the presentation. We therefore advice others to provide one or two empty slides in the beginning of the presentation if the audience is new to the software, so the application and its functionality can be explored. Since some users mixed textual content with links in the early tests, we now separated media upload, link sharing and questions (or other text) from each other.
The reaction-system was the one implemented last, which is why we are still missing more thorough insight into how listeners use it. In informal feedback rounds about the mechanism, the usage of emoji seemed to be understood well, however, these conversations were held with digital professionals and will require further analysis.
The voting mechanism was also understood by users and no questions have yet arisen from it. To our surprise this functionality, like the possibility of having different paths through the presentation, did not seem to impress or excite users too much.

The biggest weakness of unveil for most users, was the inability for the presentation to be permanently altered. Especially in meetings and informational presentations, many listeners asked for a link to the collaboratively created presentation afterwards, some users were forced to reload the website due to cross-browser compatibility issues and lost the current state of the slides, late-comers also did not have the possibility to jump into an already altered presentation. This will make it necessary to create a more intelligent and opinionated back end in the future.

\section{Creation of Presentation}
Most of the feedback we have collected about unveil over the last months came from listeners. This has two reasons: Firstly, the creation of presentations requires knowledge and experience with front end web development technologies, secondly, we have not started promoting the resulting libraries yet, as we feel the system is not stable and mature enough to be used outside. However, external developers have provided us with their feedback regarding the syntax used for defining slide decks and seem to not have had any problems understanding the usage of the libraries, what validates our decision to choose semantically-named tags rather than HTML class names to identify different components. On the downside, we seem to have overestimated the level of knowledge necessary to create own presentations, as a few developers were not entirely familiar with the new ES6 syntax and the process of bundling JavaScript and CSS files, so an easier way of importing all necessary libraries should be provided in an example presentation.
In the long run, this product will only be able to increase its popularity if a visual editor for authoring slides, as well as a system to manage (i.e. host) them will be available. Some users also raised the question of how and if it was possible to import PowerPoint presentations, to be able to use the created mechanisms in connection with already existing software.

\section{Architecture}
Generally, it has to be said that the project is still only a prototype and does not provide the stability necessary for us to feel confident promoting the product. One particular problem we have been experiencing irregularly is the navigation getting stuck in an infinite redirect loop when having more than one presenter at a given time, who try to navigate simultaneously. 
Although the chosen architecture allows for a lot of flexibility and freedom for developer and in that sense fulfils the criteria impress.js and reveal.js did not, we have lately discovered more powerful, streamlined and widespread patterns when working with React. Since we had no experience with React prior to the start of the development, best practices oftentimes only became apparent throughout the project and through the work on and with other React applications. To make the libraries yet easier to use and extend and more easily debuggable, future iterations of unveil will be based on redux \cite{redux}. This will make it easier to decouple components from each other through the introduction of a state container.
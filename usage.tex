\chapter{Usage of Resulting Library}
\label{cha:usage}

\section{Example Application}
\label{sec:implementation-client}
% how is everything defined? what has to be included?
% what are the steps of building an application with unveil?
% how are the slides defined? how are they styled? what about the modes?

To conclude this chapter, I want to show how all the discussed libraries in the end can be put together to a presentation. The whole presentation can be found in the unveil-client-server repository.

The first step to use unveil.js and its extensions is to start a new project, require the necessary npm packages and set up an \texttt{index.html} page which includes the necessary stylesheets and scripts.
The entry point for these scripts is the file \texttt{index.js}, in which the whole presentation is set up and unveil.js is configured by setting up its modes (see program \ref{prog:implementation-client-modes}). As explained before, slides are defined in HTML using the components unveil.js and its extensions provide. An example of this is shown in program \ref{prog:implementation-client-presentation}.

\begin{program}
\caption{Mode definition for setting up an unveil.js presentation. Speaker and projector modes are omitted to keep the example short but follow the same pattern as the default mode.}
\label{prog:implementation-client-modes}
\begin{JsCode}
const modes = {
  default: {
    controls : [
      KeyControls, TouchControls, UIControls,
      NavigationReceiver,
      MediaSender, MediaReceiver,
      VotingNavigatableSetter, VotingReceiver
    ],
    presenter: Presenter
  },
  speaker: {...},
  projector: {...}
};
\end{JsCode}
\end{program}

\begin{program}
\caption{Creation of presentation using modes from program \ref{prog:implementation-client-modes}. Sets up two slides as an example. The DOM will be attached to the element of id \texttt{unveil} in the base HTML document.}
\label{prog:implementation-client-presentation}
\begin{JsCode}
ReactDOM.render((
  <UnveilApp modes={modes}>
    <Slide name="start">
      <h1>Unveil</h1>
      <h2>a meta presentation</h2>
    </Slide>
    <Slide name="problem">
      <img src="./img/problem.png" />
      <Notes>someone always wanted to show something from their device</Notes>
    </Slide>
    ...
  </UnveilApp>
), document.getElementById('unveil'));
\end{JsCode}
\end{program}

The styling of the presentation is handled by CSS. The \texttt{Slide} component automatically adds the name of a slide as its id, allowing to efficiently style slide by slide, as well as applying styles for all slides at once using the \texttt{.slide} class. Program \ref{prog:implementation-client-styles} shows an example of how to style slides.

\begin{program}
\caption{Example styling for slides using Sass. In this particular piece of code, the font family of all slides is set and a background image is added to the start slide.}
\label{prog:implementation-client-styles}
\begin{CssCode}
  .slide
    font-family: 'Open Sans'
  #start
    background-image: url('../img/explore.jpg')
\end{CssCode}
\end{program}

\begin{program}
\caption{Shortened version of \texttt{NavigationReceiver}. First the inherited context properties are set up, then an observable waiting for \texttt{state:change} events from the socket is created. If the incoming request is not the currently displayed slide, the navigator will be pushed a new value.}
\label{prog:usage-network-sync-navigation-receiver}
\begin{JsCode}
// imports...

export default class NavigationReceiver extends React.Component {
  static contextTypes = {
    navigator:   React.PropTypes.object.isRequired,
    routerState: React.PropTypes.object.isRequired
  }

  componentDidMount () {
    this.observable = Observable.fromEvent(SocketIO, 'state:change')
      .filter((e) => !this.context.routerState.indices.equals(e))
      .subscribe(this.props.navigator.next)
  }
  ...
}
\end{JsCode}
\end{program}
%

\begin{program}
\caption{Example code for preparing a slide with voting.}
\label{prog:implementation-interactive-structure}
\begin{JsCode}
<Voting name="like">
  <Question>Do you like these slides?</Question>
  <Answer value="yes">Yes</Answer>
  <Answer value="no">No</Answer>
</Voting>
\end{JsCode}
\end{program}
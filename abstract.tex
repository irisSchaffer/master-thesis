\chapter{Abstract}
Mobile devices such as smartphones, tablets and laptops have become our every day companions and can act as an endless source of information, knowledge and inspiration. However, despite research showing the benefits targeted smartphone use can have in classrooms and meetings, they are still perceived as rude and disturbing during presentations. These, on the other hand, are often still one-man endeavours, from slide preparation, giving the talk to lastly follow-up work such as hand-outs. In an effort to foster listener-listener and listener-speaker interaction and to make presentations more memorable, engaging, and collaborative, a prototype of an interactive, customisable, extensible, web-based presentation platform was developed. The implemented mechanisms are the result of an analysis of several types of presentations and their weaknesses and, amongst others, include the possibility for the audience to browse and follow slides on any mobile device, instantaneously react to content with emoji and vote on polls (both prepared and created on-the-fly). Moreover, the functionality to alter the $2$-dimensional slide-sets in real time was realised, giving audience members the opportunity to actively shape a presentation by sharing multi-media content as new main or sub-slides.

Although more thorough, long-term studies will be necessary to validate our approach, consequent informal evaluation of the system in internal presentations and meetings has been promising and decidedly positive. Despite showing room for improvement, all features were understood by the users, with the content sharing functionality sparking most interest and excitement among listeners. The observation of the usage of the tool has moreover given way to further research projects and ideas. 
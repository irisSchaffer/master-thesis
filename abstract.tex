\chapter{Abstract}
Mobile devices such as smartphones, tablets and laptops have become our every day companions and can act as an endless source of information, knowledge and inspiration. However, despite studies having demonstrated the benefits of targeted smartphone use in classrooms and meetings, they are still perceived as rude and disturbing during presentations. These, on the other hand, are often still one-man endeavours, from slide preparation, giving the talk to lastly follow-up work such as hand-outs. In an effort to destigmatise mobile devices usage and make presentations more memorable, engaging, and collaborative, a prototype of an extensible JavaScript presentation eco\-system with a multitude of interactive mechanisms was implemented. These functionalities are the result of the analysis of several types of presentations and their weaknesses and, amongst others, include the possibility for the audience to browse and follow slides on any mobile device to account for individual learning-pace, as well as spontaneous reactions via emoji and votes on polls (both prepared and created on-the-fly) to more reliably estimate ones crowd's mood and background knowledge. Moreover, to truely involve audience memers in shaping the presentation, the functionality to alter the $2$-dimensional slide-sets in real time was realised. This way, listeners can share related multi-media content as well as questions and comments as new main or sub-slides.

Although more thorough, long-term studies will be necessary to validate our approach, consequent informal evaluation of the system in internal presentations and meetings has been promising and decidedly positive. Despite showing room for improvement, all features were understood by the users, with the content sharing functionality sparking most interest and excitement among listeners. The observation of the usage of the tool has moreover given way to further research projects and ideas. 
\chapter{Conclusions}
\label{cha:conclusion}

In this thesis, we present \emph{unveil}: an extensible JavaScript presentation eco\-system with a multitude of interactive mechanisms, aiming to make presentations more engaging, memorable and collaborative. At its core stand four React libraries, connecting presenters, listeners and projectors through a WebSocket server, making it possible for both audience and speakers to alter $2$-dimensional slide-sets in real time.
Different types of presentations were analysed to find their weaknesses and establish ways of enhancing the presentation experience: To make it easier to esitmate the listeners' level of knowledge, a real-time voting mechanism was implemented, allowing for both prepared polls and ones created on-the-fly during the presentation. Furthermore, the audience can instantaneously react to slides via emoji, giving the speaker an impression of the current mood. To account for individual learning-pace and late-comers, members of the audience can moreover browse and follow the slides on their personal devices as well as send questions to the presenter.
These questions, along with other multi-media content, such as pictures and links to websites and youtube videos, can then be embedded into the presentation as new main or sub-slides, effectively enabeling the audience to truely shape he presentation using nothing but the mobile devices they carry on them. This holds the potential of engaging listeners in the presentation, steering it towards certain topics, connecting members of the audience with each other and the speaker and adding related content for further reference.

Despite making presentations more memorable for the audience, this approach poses new challenges and requires more flexibility from the speaker, which is why all realised mechanisms can easily be activated an deactivated. Since the created libraries were generally designed for other developers to re-use, modify and extend, we moreover offer ways of tailoring and configuring every last detail, from the routing logic over the emoji displayed to the mechanisms enabled for each user group (presenter, listener, projector).

Although long-term studies will be necessary to validate our approach and verify and quantify their success, initial observation and evaluation of the system showed promising results. The users understood the interface and were especially excited about the possibility of sharing their own content and thoughts with the presentation. This mechanism was particularly well-accepted during informal meetings, were the usage organically evolved into a brain-storming like activity, involving all participants. The initial evaluation of our prototype also showed room for improvement, amongst others, the persistence of the created presentation and the absence of native sharing features on the phone. In future iterations of the libraries, the stability and testability of the platform will be improved as well as offering a way of permanently altering the presentation.
Moreover, a graphical user interface for the creation of slides is desirable, effectively opening interactive presentations to non-developers. The trials also gave way to ideas for future projects, most notably the combination of these interactive mechanisms with existing presentation software, with canvas-based ones being of special interest.
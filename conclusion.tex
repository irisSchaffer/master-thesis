\chapter{Conclusions and Future Work}
\label{cha:conclusion}

In this thesis, we present \emph{unveil}: an extensible JavaScript presentation eco\-system with a multitude of interactive mechanisms, aiming to make presentations more engaging, memorable and collaborative. At its core stand four React libraries, connecting presenters, listeners and projectors through a WebSocket server, making it possible for both audience and speakers to alter $2$-dimensional slide-sets in real time.
Different types of presentations were analysed to find their weaknesses and establish ways of enhancing the presentation experience: To make it easier to esitmate the listeners' level of knowledge, a real-time voting mechanism was implemented, allowing for both prepared polls and ones created on-the-fly during the presentation. Furthermore, the audience can instantaneously react to slides via emoji, giving the speaker an impression of the current mood. To account for individual learning-pace and late-comers, members of the audience can moreover browse and follow the slides on their personal devices as well as send questions to the presenter.
These questions, along with other multi-media content, such as pictures and links to websites and youtube videos, can then be embedded into the presentation as new main or sub-slides, effectively enabeling the audience to truely shape he presentation using nothing but the mobile devices they carry on them. This holds the potential of engaging listeners in the presentation, steering it towards certain topics, connecting members of the audience with each other and the speaker and adding related content for further reference.

Despite making presentations more memorable for the audience, this approach poses new challenges and requires more flexibility from the speaker. For this reason, all realised mechanisms can easily be activated an deactivated by the creator of the presentation. Since the created libraries were generally designed for other developers to re-use, modify and extend, we moreover offer ways of tailoring and configuring every last detail, from the routing logic over the emoji displayed to the mechanisms enabled for each user group (presenter, listener, projector).

\section{Future Work}
\label{sec:discussion-future-work}

Although long-term studies will be necessary to fully evaluate our approach and verify and quantify their success, the early user study of the system showed promising results. The users understood the interface and were especially excited about the voting mechanism. The initial evaluation of our prototype also showed room for improvement, amongst others, the persistence of the created presentation as well as the missing graphical user interface to create presentations. In future iterations of the libraries, the stability and testability of the platform will be improved as well as offering a way of permanently altering the presentation.

For the presenter to profit even more from the new possibilities of interactive presentations, we would like to provide some analytics: When did the audience interact with the presentation most? When exactly were which reactions triggered? Which slides provoked the most input or questions and how long did audience members stay on each slide when browsing through them individually?

As far as the interactive mechanisms implemented in this project are concerned, initial study has shown potential for improvement as well: the content sharing tool was used seldomly and the wish for a commenting or annotating functionality was mentioned by some users. Votings could offer several types (e.g. multiple choice, open questions, ratings), as well as different ways of visualising the results (e.g. pie charts, stars, clusters of answers). They could moreover be more tightly linked to paths through the presentation, so the result of a voting could directly link to a path.
Reactions generally need more testing and although we are content with the current functionality for the scope of this thesis, analytics for the presenter and time-based rather than slide-based display of reactions are desireable.
Overall, more animations could make the presentation feel even more responsive and offer a more \emph{native} feel for mobile users \cite{GoogleMaterialDesignGuide}.

From an architectural point of view, it is our declared objective to deliver a stable first version of the unveil ecosystem within the coming months. Redux and even further separation of concerns will make it easier to automatically test the application and add new features, such as the persistence of presentation state to local storage or a database. Moreover, porting the existing system to React Native and adding native sharing possibilities is worth a consideration.
While one of the presentations held with unveil so far has been remote, the possibilities of using this project with distributed audiences has yet to be studied. However, the opportunity of reacting to slides as well as adding content (potentially after a talk or meeting) are a promising foundation for remote presentations \cite{Isaacs:InteractivePresentationsDistributedAudience, Cheng:TreebasedOnlinePresentations} and should be examined further.

A possibility not at all touched in the present work is the development of an authoring and hosting tools for unveil presentations. This would make it easier to create slide decks and render the necessity for front end development knowledge obsolete; effectively giving anyone a tool to create interactive presentations. This would also be possible offering imports of PowerPoint presentations or even re-building the interactive mechanisms as PowerPoint plugins.

Another opportunity which arose from the users' desire to share to a blank presentation during meetings as well as questions about annotations, would be to follow a canvas-based rather than a slide-based approach in future projects. That way meeting participants could effectively use the platform as a tool of collaboratively creating and sharing content, to brainstorm and take notes from any device in a shared working space.

Finally, more formal observations and long-term studies will be necessary to quantify the success of the developed mechanisms. Currently, the tool is mostly in use for informal Monday morning presentations with other digital professionals; the acceptance and usability of the application will have to be re-visited and re-evaluated when assessed with less technologically versed users.
\chapter{Related Work}
\label{cha:related-work}

The idea of using electronic devices to foster group interaction in meetings is not a new one. Stefik et al. \cite{Stefik:BeyondTheChalkboard} already experimented with the use of personal computers in meeting rooms as early as 1987. Since then, digital whiteboards, telepresence systems, productive multi-user web applications and other computer-aided collaboration tools have become a common sight. The use of portable devices like clickers or mobile phones in presentations, however, has only been studied over the last 5 to 10 years.

\subsection{Classroom-related}
Most of this research has been conducted in the educational sector, where growing class-sizes have caused student participation to sink drastically \cite{Bry:Backstage}. The first ap\-proa\-ches in this field of student-response-systems (SRS) utilised so-called clickers -- remote-control-like devices, connected to a receiver station via radio frequency technology \cite{cuclickers:faq} which can be used for tasks like taking attendance and voting \cite{Chamillard:StudentResponseSystem}. Using these clicker systems has shown to ``yield a strong and positive relationship with student learning'' \cite{Chamillard:StudentResponseSystem}. However, the limitations of clickers -- the need for proprietary hardware and the limited interface consisting only of a few buttons -- lead researchers to experiment with personal mobile devices as input instead. In 2007 Lindquist et al. \cite{Lindquist:ExploringMobilePhonesActiveLearning} presented a system integrated with the University of Washington's Classroom Presenter software, which lets students submit answers to assignments and in-class quizzes via SMS and MMS or using their laptops. Although the mobile phone users struggled with the input of longer messages, they perceived the ubiquity and concenience of using a light-weight personal device as an advantage. Most students, however, were worried about the costs of using SMS or MMS as a requirement in class -- a concern modern devices with internet access and cheap data plans dispel. The first of these web-based approaches were explored around the same time. Esponda \cite{Esponda:ElectronicVotingOnTheFly} for example describes a system in which iPods and other devices with access to wifi can be used to answer questions during class. What is interesting about her approach is not only the technology used, but also that questions do not have to be prepared in advance, but can also be created on-the-fly, using a pen-based tablet, resulting in more lively and spontaneous student-teacher-interaction.

More recent publications make use of modern web browsers' possibilities to enhance lectures and the interactivity of presentations. The tool \emph{ASQ} \cite{Triglianos:InteractiveWebPresentationsImpress} for example lets lecturers create HTML5 presentations with \emph{impress.js}\footnote{http://impress.github.io/impress.js} which are then distributed to listeners via a link. Students follow the presentations on their mobile devices, and can submit questions connected to the current slide to the speaker. Quizzes (both open questions and multiple-choice) can be embedded in the slides by the teacher. These quizzes can either be graded automatically (for coding assignments and multiple-choice questions), corrected by teaching assistants or by the students in self or peer-assessment. While this project has put a lot of effort into the server-side and administration of slidesets, the present approach concentrates more on the client-side and does not provide management tools. However, as noted by Esponda \cite{Esponda:ElectronicVotingOnTheFly}, being familiar with the listeners' understanding of a subject is important for creating the polls, which is why we also added the possibility of creating votings spontaniously.
Another interesting approach is presented by Cheng et al. \cite{Cheng:TreebasedOnlinePresentations}, who propose a system which generates HTML presentations from \emph{Microsoft PowerPoint} slides and lets viewers add their own content (either additional material or questions) as vertical sub-slides to them. This way a tree-like structure is created in which teachers and students collaborate in interactive presentations. This architecture also inspired the sub-slide based presentation space deployed in the present work.

Another popular application, with richer audience-spea\-ker-in\-ter\-ac\-tion and an emphasis on listener-listener-interaction is \emph{Backstage}\footnote{http://backstage.pms.ifi.lmu.de/}. As digital backchannels like Twitter can foster the sense of community within the audience, but are usually hard to follow for presenters, Bry et al.\cite{Bry:Backstage} developed a backchannel specifically for large classrooms. Students can post messages publicly and send private messages to their colleagues. These public posts can be up or down-voted, as well as marked as unrelated. Together with an ageing-algorithm, this community feedback is used to estimate a post's relevance. Important feedback is then presented to the lecturer, to allow him or her to get a better sense for the audiences' opinion and understanding. Additionally, small quizzes and polls serve as performance feedback to the teacher and students.

\subsection{Meeting-environments and general approaches}

In contrast to classroom-related software, meeting-en\-vi\-ron\-ments usually have an significantly lower amount of participants, as well as a smaller gap between the speaker and the audience. Another difference lies in the polling, surveying and quizzing functionality which most of the presented projects offer: while these usually have one correct answer in educational settings, to grade students \cite{Lindquist:ExploringMobilePhonesActiveLearning, Triglianos:InteractiveWebPresentationsImpress, Bry:Backstage}, the goal in other environments is to make decisions and collect ideas, without judgment.
One publication which concentrates on these polls and their real-time evaluation and rendering is \cite{Inoue:RealTimeQuestionnaire}: Inoue et al. present a system which distributes \emph{Microsoft PowerPoint} presentations using modern web-technologies while making it possible to alter and update the slides in presentation mode. This way questionnaires can be answered and their results displayed in real-time. Additionally, members of the audience can add annotations (both handwritten and digital) to slides. Although this approach seems very promising, pictures, videos and other types of media are ignored entirely. Moreover the interface seems to complicated to be used on small devices and is therefore only usable on laptops and maybe tablets. Systems designed for multi-device use are proposed in \cite{Bohmer:SmartphoneUseRude} and \cite{Teevan:MobileFeedbackDuringPresentation}: In \cite{Bohmer:SmartphoneUseRude} as well as examining the perception of smartphone use in meetings, the mobile application \emph{Meetster} is presented. The study finds that although people primarily use their phones for meeting or work-related tasks, they tend to think their colleagues use theirs for private purposes. Unlike the present thesis, in which mobile devices should be used in the context of presentations, \emph{Meetster} was developed to help getting to know other meeting attendees in a playful way. This changed the perception of using one's smartphone during the meeting and was described as ``fostering social interactions''.

\begin{figure}
  \includegraphics[width=0.9\columnwidth]{feedback-meetings-ms-research}
  \caption{\emph{Crowd Feedback} \cite{Teevan:MobileFeedbackDuringPresentation} used during a presentation. The bar next to the slides shows one dot per participant in the meeting. The feedback dots fade out over time.}
  \label{fig:related-work-crowd-feedback}
\end{figure}

\emph{Crowd Feedback} \cite{Teevan:MobileFeedbackDuringPresentation}, on the other side, is a system for displaying continuous, real-time feedback to the speaker in presentations. A responsive web application with a like and dislike button controls the feedback-system. The participants' reactions are shown with a red (dislike) or green (like) dot for each attendee in a sidebar next to the presentation slides (see figure \ref{fig:related-work-crowd-feedback}). An evaluation of the system showed that the participants felt more engaged with the presentation and connected to other listeners. Many users stated only having the possibility to like or dislike did not reflect enough options and that a button related to the speech pace might have helped. It was also noted that the sidebar was perceived as disturbing and made it harder to pay close attention to the presentation.

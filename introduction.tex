\chapter{Introduction}
\label{cha:introduction}

\section{Motivation}

Mobile phones, tablets and laptops have become our every day companions. We take them with us wherever we go, may it be the classroom or meetings, lately they have even made an appearance in courtrooms \cite{Farrell:TrialByTablet}. Especially during presentations mobile device usage is still perceived as impolite and can be a source of distraction \cite{Bohmer:SmartphoneUseRude, Bajko:ComparativePerceptionSmartphoneMeeting, Kuznekoff:ImpactPhoneStudentLearning}, although studies indicate that lecture-relevant phone use in classrooms can actually be beneficial for information-recall \cite{Kuznekoff:MobilePhoneClassroomTwitter}.

Presentations, on the other hand, have remained largely unchanged since the launch of PowerPoint in the late 1980s \cite{Yates:PowerPoint}. In fact, some of the features overhead projectors innately offered, namely the annotation of transparencies during the presentation and sorting and choosing slides before presenting them, have effectively been lost with the introduction of presentation software. While the amount of presentations given has continuously risen, the modus of presenting has stayed untouched: In most cases, there is one presenter and a group of co-located listeners. The speaker prepares slides prior to a presentation and has the responsibility of educating, fascinating, inspiring and keeping the audience awake, while catering the presented contents to the respective listeners. Although interactive elements in presentations have proven to be twice as effective at engaging listeners and beneficial to information-recall \cite{prezi-science}, presentations have largely remained a static endeavor for the speaker alone.

Mobile phones, tablets and laptops, however, hold the potential of challenging this status quo. Their growing computing power as well as ubiquitousness make them suitable candidates for interacting with presentation software, thus transforming presentations into a more collaborative effort. Instead of banning modern technologies, incorporating mobile devices has proven to foster collaboration and connection between attendees in meetings \cite{Bohmer:SmartphoneUseRude} and has the potential of promoting participation and helping introverts overcome the hurdle of speaking out loud \cite{Bry:Backstage}. At the core of this thesis therefore stands the question: How can mobile devices be integrated into presentation workflows to engage and involve the audience, while transforming the stigma around mobile phones into something positive?

\section{Goals}

\begin{figure}
\centering
\includegraphics[width=.8\textwidth]{illustrations/meeting}
\caption{Unveil presentation in action: the projector displays the current slide, listeners can follow and interact with presentations on their personal devices and listeners can control the presentation using theirs.}
\label{fig:introduction-meeting}
\end{figure}

The main objective of the project consequently is the exploration of different ways of incorporating personal devices in presentations efficiently and productively. This involves both the conception of such mechanisms, as well as their implementation in an online presentation tool.
Due to the high number of different settings and contexts in which talks can be given, and since we feel meetings and other business-related presentations with small numbers of attendees offer the perfect playground and most possibilities for interactive mechanisms, this thesis focuses on presentations in business-settings.

The proposed presentation software includes device-independent altering of pre-defined $2$-dimensional slide-sets by listeners and presenters in real time. It allows audience members to view slides on their personal devices, either navigating freely or synchronised with the presenter (see figure \ref{fig:introduction-meeting}). Additionally, the developed libraries offer support for real-time voting, as well as voting creation on-the-fly. Instantaneous audience reactions via emoji and different paths through the presentation were also realised.

As far as the implementation is concerned, it was our declared goal to create a modular ecosystem other developers can tap into, reuse, overwrite and extend. Since the web was chosen as a platform for its rapid prototyping and iteration cylce possibilities, it was paramount to make the application feel as fast and responsive as possible, to give it the look and feel of a native application \cite{Charland:WebVsNative}. Therefore the user interface and interaction design were at the core of the project with the overall objective of creating an interface that works across all devices, without feeling unnatural.

Finally, the developed mechanisms were tested and evaluated in an early user study by a group of digital marketing professionals.

In total, our contribution can be summarised as follows:
\begin{itemize}
\item Analysis of presentations, finding design recommendations for interactive mechanisms
\item Implementation of presentation software following these recommendations
\item Study and evaluation of the implemented software
\item Creation of modular, open-source ecosystem which can  be used and extended by other developers
\end{itemize}

\section{Structure}

This thesis is organised into eight main chapters. First of all, we want to establish the context around the present work by introducing the reader to existing research and projects in the field of interactive presentations in chapter \ref{cha:related-work}. Since an overwhelming number of studies have been conducted in educational settings, the chapter is further divided into classroom (section \ref{sec:related-work-classroom}) and office (section \ref{sec:related-work-office}) related approaches as well as general presentations (section \ref{sec:related-work-general}).

In chapter \ref{cha:mechanisms}, different types of presentations are analysed for their shortcomings and suitable solutions are developed. These are formulated into distinct interactive mechanisms for which clear requirements are established.
Based on these requirements, chapter \ref{cha:design} goes into details on the interface and interaction design of the application as a whole and each of its interactive mechanisms. The general flow and setup of the proposed software are also discussed.

Chapter \ref{cha:implementation} lays out the entire implementation of the libraries involved in the presentation tool. It first defines the scope of the project (section \ref{sec:implementation-scope}) to then cover the server setup (section \ref{sec:implementation-server}), before offering more insight into the underlying front end technologies used (section \ref{sec:implementation-technologies}). Finally the general project structure (section \ref{sec:implementation-structure}), as well as all developed libraries are discussed in detail.

Chapter \ref{cha:results} looks at the results of our work from a user's and developer's perspective. First we discuss the final application and present screenshots of it, to then talk about the conducted user study. Since we pride ourselves in having developed an entirely open-sourced presentation ecosystem which we want others to explore, reuse and extend, we then shortly address the usage of the resulting libraries from a developer's perspective.

Chapter \ref{cha:discussion} outlines the results of our informal evaluation and observation of the system in regards to its usability \ref{sec:discussion-usability} and the creation of presentations \ref{sec:discussion-dev}, to then reflect on the chosen architecture \ref{sec:discussion-architecture}.
Finally, we briefly talk about the conclusions we draw from this project in chapter \ref{cha:conclusion} and  give an outlook on future work.

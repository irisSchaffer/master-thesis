\chapter{Introduction}
\label{cha:introduction}

\section{Introduction}


\section{Goals}
The main aim of the project and thesis is to explore different ways of incorporating mobile devices into presentations in business-settings efficiently and productively. This includes polls created by the speaker (both beforehand and on-the-fly), as covered by other studies, as well as continuous, spontaneous feedback as described in \cite{Teevan:MobileFeedbackDuringPresentation}. Additionally other types of annotations should be supported, namely textual comments, images, links and youtube-videos, which will be rendered accordingly on new presentation slides. I believe this approach has the potential of transforming presentations into an collaborative effort in which all meeting-participants are empowered to shape the progress and outcome of presentations. From a technological point of view, like most existing approaches \cite{Bry:Backstage, Cheng:TreebasedOnlinePresentations, Esponda:ElectronicVotingOnTheFly, Inoue:RealTimeQuestionnaire, Teevan:MobileFeedbackDuringPresentation, Triglianos:InteractiveWebPresentationsImpress}, a web application will be developed to make use of modern web technologies' quick prototyping capabilities and the web's cross-platform and cross-device nature. As \emph{WebSockets} have successfully been deployed in the real-time features of other presentation tools \cite{Inoue:RealTimeQuestionnaire, Triglianos:InteractiveWebPresentationsImpress} (in \cite{Inoue:RealTimeQuestionnaire} a difference in response time between 10 and 600 simultaneous users of under 150ms and a package loss of approximately 0\% was reported), this technology will be used to communicate between speaker and audience. Like in \cite{Triglianos:InteractiveWebPresentationsImpress}, an existing web presentation library will be used to be able to concentrate on the collaborative features rather than building a cross-device presentation platform. Instead of \emph{impress.js}, \emph{reveal.js}\footnote{http://lab.hakim.se/reveal-js/} will be used for this purpose, as it already offers a few key features, namely focus on mobile devices, embedded videos, speaker-notes and a possibility to follow or control presentation from ones personal device.
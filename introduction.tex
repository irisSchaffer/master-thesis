\chapter{Introduction}
\label{cha:introduction}

\section{Motivation}

Mobile phones, tablets and laptops have become our every day companions. We take them with us wherever we go, may it be the classroom or meetings, lately they have even made an appearance in courtrooms \cite{Farrell:TrialByTablet}. Especially during presentations mobile device usage is still perceived as rude and can be a source of distraction \cite{Bohmer:SmartphoneUseRude, Bajko:ComparativePerceptionSmartphoneMeeting, Kuznekoff:ImpactPhoneStudentLearning}, although studies indicate that lecture-relevant phone use in classrooms can actually be beneficial for information-recall \cite{Kuznekoff:MobilePhoneClassroomTwitter}.

Presentations, on the other hand, have remained largely unchanged since the launch of PowerPoint in the late 1980s \cite{Yates:PowerPoint}. In fact, some of the features overhead projectors innately offered, namely the annotation of transparencies during the presentation and sorting and choosing slides before presenting them, have effectively been lost with the introduction of presentation software. While the amount of presentations given has continuously risen, the modus of presenting has stayed untouched: In most cases, there is one presenter and a group of co-located listeners. The speaker prepares slides prior to a presentation and has the responsibility of educating, fascinating, inspiring and keeping the audience awake, while catering the presented contents to the respective listeners. Although interactive elements in presentations have proven to be twice as effective at engaging listeners and beneficial to information-recall \cite{prezi-science}, presentations have largely remained a static endeavor for the speaker alone.

Mobile phones, tablets and laptops, however, hold the potential of challenging this status quo. Their growing computing power as well as ubiquitousness make them suitable candidates for interacting with presentation software, thus transforming presentations into a more collaborative effort. Instead of banning modern technologies, incorporating mobile devices has proven to foster collaboration and connection between attendees in meetings \cite{Bohmer:SmartphoneUseRude} and holds the potential of promoting participation and helping introverts overcome the hurdle of speaking out loud \cite{Bry:Backstage}. At the core of this thesis therefore stands the question: How can mobile devices be integrated into presentation workflows to engage and involve the audience, while transforming the stigma around mobile phones into something positive?

\section{Goals}

The main aim of the project consequently is the exploration and conception of different ways of incorporating personal devices in presentations efficiently and productively. Due to the high number of different settings and contexts in which talks can be given, and since we feel meetings and other business-related presentations with small numbers of attendees offer the perfect playground and most possibilities for interactive elements, this thesis concentrates on presentations in business-settings.




\section{Structure}

\chapter{Kurzfassung}

\begin{german}
Mobile Geräte wie Smartphones, Tablets und Laptops sind unsere täglichen Wegbegleiter und dienen als endlose Quellen an Information, Wissen und Inspiration. Obwohl Studien die positiven Effekte von zielgerichtetem Einsatz von Smartphones in Klassenzimmern und Meetings bewiesen haben, gilt die Verwendung dieser in Präsentationen noch immer als störend und unhöflich. Präsentationen, andererseits, sind von der Vorbereitung der Folien, dem Vortragen bis hin zur Nachbereitung von Hand-outs meist immer noch die alleinige Aufgabe der Präsentatoren. In einem Versuch die Publikum-Publikum sowie Publikum-Präsentator-Interaktion zu stärken und Präsentationen ansprechender, einprägsamer und kollaborativer zu gestalten, wurde ein Prototyp einer interaktiven, konfigurierbaren und erweiterbaren, web-basierten Präsentationsplatform entwickelt. Die implementierten Mechanismen bauen auf der Analyse unterschiedlicher Präsentationsformen und ihrer Schwächen auf und umfassen unter anderem die Anzeige und Synchronisation der Folien auf persönlichen Geräten, momentane Reaktionen auf Präsentationsinhalte mit Emoji und das Abstimmen auf vorbereitete und während der Präsentation angelegte Umfragen. Des Weiteren können die $2$-dimensionalen Foliensätze in Echtzeit verändert werden, was es dem Publikum ermöglicht die Präsentation aktiv, durch das Teilen multimedialer Inhalte, zu gestalten.

Obwohl tiefergehende Langzeitstudien notwendig sind um unseren Ansatz zu legitimieren, sind unsere ersten informellen Evaluierungen des Systems in internen Meetings und Präsentationen vielversprechend und durchwegs positiv. Auch wenn unsere Auswertungen Verbesserungsmöglichkeiten verdeutlicht haben, so wurden doch alle Mechanismen von den Benutzer/innen verstanden, und besonders das Teilen multimedialer Inhalte weckte Interesse bei den Zuhörer/innen und konnte diese begeistern. Die Beobachtung der Verwendung der geschaffenen Werkzeuge konnte außerdem neue Ideen inspirieren und mögliche weitere Forschungsprojekte aufzeigen.
\end{german}
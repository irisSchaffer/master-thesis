\chapter{Kurzfassung}

\begin{german}
Mobile Geräte wie Smartphones, Tablets und Laptops sind unsere täglichen Wegbegleiter und dienen als endlose Quellen an Information, Wissen und Inspiration. Obwohl Studien die positiven Effekte von zielgerichtetem Einsatz von Smartphones in Klassenzimmern und Meetings bewiesen haben, gilt die Verwendung dieser in Präsentationen noch immer als störend und unhöflich. Präsentationen, andererseits, sind von der Vorbereitung der Folien, dem Vortragen bis hin zur Nachbereitung von Hand-Outs meist immer noch die alleinige Aufgabe der Präsentatoren. In einem Versuch das Stigma um mobile Geräte zu wenden und Präsentationen ansprechender, einprägsamer und kollaborativer zu gestalten, wurde der Prototyp einer erweiterbaren JavaScript Präsentationsumgebung mit mehreren interaktiven Mechanismen implementiert. Die entwickelten Funktionen bauen auf der Analyse unterschiedlicher Präsentationsformen und ihrer Schwächen auf und umfassen unter anderem die Anzeige und Synchronisation der Folien auf persönlichen Geräten um auf individuelles Lernverhalten einzugehen, sowie momentane Reaktionen auf Präsentationsinhalte mit Emoji und das Abstimmen auf vorbereitete und während der Präsentation angelegte Umfragen, um das Abschätzen des Vorwissens und der Stimmung des Publikums zu erleichtern. Des Weiteren werden Zuhörer_innen aktiv in die Gestaltung der Präsentation miteinbezogen indem es ihnen ermöglicht wird Fragen, Kommentare und andere multimediale Inhalte als neue Haupt- und Subfolien mit der Präsentation zu teilen.

Obwohl tiefergehende Langzeitstudien notwendig sind um unsere Ansätze zu legitimieren, sind unsere ersten informellen Evaluierungen des Systems in internen Meetings und Präsentationen vielversprechend und durchwegs positiv. Auch wenn unsere Auswertungen Verbesserungsmöglichkeiten verdeutlicht haben, so wurden doch alle Mechanismen von den Benutzer_innen verstanden, und besonders das Teilen multimedialer Inhalte weckte Interesse bei den Zuhörer_innen und konnte diese begeistern. Die Beobachtung der Verwendung der geschaffenen Werkzeuge konnte außerdem neue Ideen inspirieren und hat mögliche weitere Forschungsprojekte aufgezeigt.
\end{german}
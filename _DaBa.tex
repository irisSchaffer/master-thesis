%% File encoding: UTF-8
%% äöüÄÖÜß  <-- keine deutschen Umlaute hier? UTF-faehigen Editor verwenden!

\documentclass[master,english]{hgbthesis}
\usepackage{pdfpages}
% Zulässige Class Options: 
%   Typ der Arbeit: diplom, master (default), bachelor, praktikum 
%   Hauptsprache: german (default), english
%%------------------------------------------------------------

\RequirePackage[utf8]{inputenc}		% remove when using lualatex oder xelatex!

\graphicspath{{images/}}    % name of directory containing the images
\logofile{logo}							% name of logo-PDF in images/ (or use \logofile{} for no logo)
\bibliography{literature}  	% name of the BibTeX (.bib) file


%%%----------------------------------------------------------
\begin{document}
%%%----------------------------------------------------------

% Einträge für ALLE Arbeiten: --------------------------------
\title{Mobile Device Usage in Interactive, Co-located Presentations}
\author{Iris M.\ Schaffer}
\studiengang{Interactive Media}
\studienort{Hagenberg}
\abgabedatum{2016}{09}{26}	% {YYYY}{MM}{DD}

%%%----------------------------------------------------------
\frontmatter
\maketitle
\tableofcontents
%%%----------------------------------------------------------

\chapter{Acknowledgments}
At this point, I would like to express my gratitude to everyone who has joined me on the venture of my masters degree, may it be actively or passively. Those who offered advice and guidance when in doubt, those who motivated and inspired me to always strive for more and those who listened to me and let me spark their interest. In particular, my thank goes to my parents who have always supported me in whatever it was I set my mind on, both financially and with advice. My friends, for both sitting me down in front of my computer when I lost motivation and for taking my mind off work when sinking into it too deeply. Special thanks to Leandro Ostera and his enthusiasm for lifting my spirits, but also for all the theoretical as well as practical advice and involvement in this project.

Moreover, I want to express my gratitude to my supervisor Dr. Michael Haller for providing me with ideas, feedback and the necessary flexibility to write this thesis remotely, from Sweden. On this note, I want to also thank Oakwood Creative AB for supporting their employee's dreams and visions and the possibility for me to test my presentation software in meetings as well as Monday morning presentations. Thanks also to my colleagues here, for their constant feedback and patience with my Swedish, as well as to every last stranger I had the pleasure of telling about my research.

Finally, my admiration and appreciation go out to all developers and companies driving the course of open-source forward. Despite mostly not scientifically publishing their results, this community has given birth to some of the most exciting and most widely used technologies, libraries and frameworks on the market. Without the work of these never-resting individuals, this thesis and project would never have come into existence in their current form. To pay tribute to this vibrant community, all the work involved in this thesis was also open-sourced and published on GitHub\footnote{\href{https://github.com/irisSchaffer?tab=repositories}{\textsf{https://github.com/irisSchaffer/}}}.
\chapter{Abstract}
Mobile devices such as smartphones, tablets and laptops have become our every day companions and can act as an endless source of information, knowledge and inspiration. However, despite studies having demonstrated the benefits  of targeted smartphone in classrooms and meetings, they are still perceived as rude and disturbing during presentations. These, on the other hand, are often still one-man endeavours, from slide preparation, giving the talk to lastly follow-up work such as hand-outs. In an effort to destigmatise mobile devices usage and make presentations more memorable, engaging, and collaborative, a prototype of an extensible JavaScript presentation eco\-system with a multitude of interactive mechanisms was implemented. These functionalities are the result of the analysis of several types of presentations and their weaknesses and, amongst others, include the possibility for the audience to browse and follow slides on any mobile device to account for individual learning-pace, as well as spontaneous reactions via emoji and votes on polls (both prepared and created on-the-fly) to more reliably estimate ones crowd's mood and background knowledge. Moreover, to truely involve audience memers in shaping the presentation, the functionality to alter the $2$-dimensional slide-sets in real time was realised. This way, listeners can share related multi-media content as well as questions and comments as new main or sub-slides.

Although more thorough, long-term studies will be necessary to validate our approach, consequent informal evaluation of the system in internal presentations and meetings has been promising and decidedly positive. Despite showing room for improvement, all features were understood by the users, with the content sharing functionality sparking most interest and excitement among listeners. The observation of the usage of the tool has moreover given way to further research projects and ideas. 			
\chapter{Kurzfassung}

\begin{german}
Mobile Geräte wie Smartphones, Tablets und Laptops sind unsere täglichen Wegbegleiter und dienen als endlose Quellen an Information, Wissen und Inspiration. Obwohl Studien die positiven Effekte von zielgerichtetem Einsatz von Smartphones in Klassenzimmern und Meetings bewiesen haben, gilt die Verwendung dieser in Präsentationen noch immer als störend und unhöflich. Präsentationen, andererseits, sind von der Vorbereitung der Folien, dem Vortragen bis hin zur Nachbereitung von Hand-outs meist immer noch die alleinige Aufgabe der Präsentatoren. In einem Versuch die Publikum-Publikum sowie Publikum-Präsentator-Interaktion zu stärken und Präsentationen ansprechender, einprägsamer und kollaborativer zu gestalten, wurde ein Prototyp einer interaktiven, konfigurierbaren und erweiterbaren, web-basierten Präsentationsplatform entwickelt. Die implementierten Mechanismen bauen auf der Analyse unterschiedlicher Präsentationsformen und ihrer Schwächen auf und umfassen unter anderem die Anzeige und Synchronisation der Folien auf persönlichen Geräten, momentane Reaktionen auf Präsentationsinhalte mit Emoji und das Abstimmen auf vorbereitete und während der Präsentation angelegte Umfragen. Des Weiteren können die $2$-dimensionalen Foliensätze in Echtzeit verändert werden, was es dem Publikum ermöglicht die Präsentation aktiv, durch das Teilen multimedialer Inhalte, zu gestalten.

Obwohl tiefergehende Langzeitstudien notwendig sind um unseren Ansatz zu legitimieren, sind unsere ersten informellen Evaluierungen des Systems in internen Meetings und Präsentationen vielversprechend und durchwegs positiv. Auch wenn unsere Auswertungen Verbesserungsmöglichkeiten verdeutlicht haben, so wurden doch alle Mechanismen von den Benutzer/innen verstanden, und besonders das Teilen multimedialer Inhalte weckte Interesse bei den Zuhörer/innen und konnte diese begeistern. Die Beobachtung der Verwendung der geschaffenen Werkzeuge konnte außerdem neue Ideen inspirieren und mögliche weitere Forschungsprojekte aufzeigen.
\end{german}		
%

%%%----------------------------------------------------------
\mainmatter         % Hauptteil (ab hier arab. Seitenzahlen)
%%%----------------------------------------------------------

\chapter{Introduction}
\label{cha:introduction}

\section{Motivation}

Mobile phones, tablets and laptops have become our every day companions. We take them with us wherever we go, may it be the classroom or meetings, lately they have even made an appearance in courtrooms \cite{Farrell:TrialByTablet}. Especially during presentations mobile device usage is still perceived as impolite and can be a source of distraction \cite{Bohmer:SmartphoneUseRude, Bajko:ComparativePerceptionSmartphoneMeeting, Kuznekoff:ImpactPhoneStudentLearning}, although studies indicate that lecture-relevant phone use in classrooms can actually be beneficial for information-recall \cite{Kuznekoff:MobilePhoneClassroomTwitter}.

Presentations, on the other hand, have remained largely unchanged since the launch of PowerPoint in the late 1980s \cite{Yates:PowerPoint}. In fact, some of the features overhead projectors innately offered, namely the annotation of transparencies during the presentation and sorting and choosing slides before presenting them, have effectively been lost with the introduction of presentation software. While the amount of presentations given has continuously risen, the modus of presenting has stayed untouched: In most cases, there is one presenter and a group of co-located listeners. The speaker prepares slides prior to a presentation and has the responsibility of educating, fascinating, inspiring and keeping the audience awake, while catering the presented contents to the respective listeners. Although interactive elements in presentations have proven to be twice as effective at engaging listeners and beneficial to information-recall \cite{prezi-science}, presentations have largely remained a static endeavor for the speaker alone.

Mobile phones, tablets and laptops, however, hold the potential of challenging this status quo. Their growing computing power as well as ubiquitousness make them suitable candidates for interacting with presentation software, thus transforming presentations into a more collaborative effort. Instead of banning modern technologies, incorporating mobile devices has proven to foster collaboration and connection between attendees in meetings \cite{Bohmer:SmartphoneUseRude} and has the potential of promoting participation and helping introverts overcome the hurdle of speaking out loud \cite{Bry:Backstage}. At the core of this thesis therefore stands the question: How can mobile devices be integrated into presentation workflows to engage and involve the audience, while transforming the stigma around mobile phones into something positive?

\section{Goals}

\begin{figure}
\centering
\includegraphics[width=.8\textwidth]{illustrations/meeting}
\caption{Unveil presentation in action: the projector displays the current slide, listeners can follow and interact with presentations on their personal devices and listeners can control the presentation using theirs.}
\label{fig:introduction-meeting}
\end{figure}

The main objective of the project consequently is the exploration of different ways of incorporating personal devices in presentations efficiently and productively. This involves both the conception of such mechanisms, as well as their implementation in an online presentation tool.
Due to the high number of different settings and contexts in which talks can be given, and since we feel meetings and other business-related presentations with small numbers of attendees offer the perfect playground and most possibilities for interactive mechanisms, this thesis focuses on presentations in business-settings.

The proposed presentation software includes device-independent altering of pre-defined $2$-dimensional slide-sets by listeners and presenters in real time. It allows audience members to view slides on their personal devices, either navigating freely or synchronised with the presenter (see figure \ref{fig:introduction-meeting}). Additionally, the developed libraries offer support for real-time voting, as well as voting creation on-the-fly. Instantaneous audience reactions via emoji and different paths through the presentation were also realised.

As far as the implementation is concerned, it was our declared goal to create a modular ecosystem other developers can tap into, reuse, overwrite and extend. Since the web was chosen as a platform for its rapid prototyping and iteration cylce possibilities, it was paramount to make the application feel as fast and responsive as possible, to give it the look and feel of a native application \cite{Charland:WebVsNative}. Therefore the user interface and interaction design were at the core of the project with the overall objective of creating an interface that works across all devices, without feeling unnatural.

Finally, the developed mechanisms were tested and evaluated in an early user study by a group of digital marketing professionals.

In total, our contribution can be summarised as follows:
\begin{itemize}
\item Analysis of presentations, finding design recommendations for interactive mechanisms
\item Implementation of presentation software following these recommendations
\item Study and evaluation of the implemented software
\item Creation of modular, open-source ecosystem which can  be used and extended by other developers
\end{itemize}

\section{Structure}

This thesis is organised into eight main chapters. First of all, we want to establish the context around the present work by introducing the reader to existing research and projects in the field of interactive presentations in chapter \ref{cha:related-work}. Since an overwhelming number of studies have been conducted in educational settings, the chapter is further divided into classroom (section \ref{sec:related-work-classroom}) and office (section \ref{sec:related-work-office}) related approaches as well as general presentations (section \ref{sec:related-work-general}).

In chapter \ref{cha:mechanisms}, different types of presentations are analysed for their shortcomings and suitable solutions are developed. These are formulated into distinct interactive mechanisms for which clear requirements are established.
Based on these requirements, chapter \ref{cha:design} goes into details on the interface and interaction design of the application as a whole and each of its interactive mechanisms. The general flow and setup of the proposed software are also discussed.

Chapter \ref{cha:implementation} lays out the entire implementation of the libraries involved in the presentation tool. It first defines the scope of the project (section \ref{sec:implementation-scope}) to then cover the server setup (section \ref{sec:implementation-server}), before offering more insight into the underlying front end technologies used (section \ref{sec:implementation-technologies}). Finally the general project structure (section \ref{sec:implementation-structure}), as well as all developed libraries are discussed in detail.

Chapter \ref{cha:results} looks at the results of our work from a user's and developer's perspective. First we discuss the final application and present screenshots of it, to then talk about the conducted user study. Since we pride ourselves in having developed an entirely open-sourced presentation ecosystem which we want others to explore, reuse and extend, we then shortly address the usage of the resulting libraries from a developer's perspective.

Chapter \ref{cha:discussion} outlines the results of our informal evaluation and observation of the system in regards to its usability \ref{sec:discussion-usability} and the creation of presentations \ref{sec:discussion-dev}, to then reflect on the chosen architecture \ref{sec:discussion-architecture}.
Finally, we briefly talk about the conclusions we draw from this project in chapter \ref{cha:conclusion} and  give an outlook on future work.

\chapter{Related Work}
\label{cha:related-work}

The idea of using electronic devices to foster group interaction in meetings and presentations is not new. Stefik et al. \cite{Stefik:BeyondTheChalkboard} already experimented with the use of personal computers in meeting rooms as early as 1987 and Myers et al. \cite{Myers:CollaborationPDAs} developed a collaboration tool which could be used to annotate PowerPoint slides from PDAs in 1998. Since then, digital whiteboards, telepresence systems, productive multi-user web applications and other computer-aided collaboration tools have become a common sight and we choose to carry smart devices around wherever we go. Surprisingly little research, however, has covered the use of these mobile devices in the context of presentations. Most of these studies were conducted in the educational sector and usually aim at quizzing students, which is why an own sub section is dedicated to classroom related approaches. While most relevant research has concentrated on one aspect, such as real-time polling \cite{Inoue:RealTimeQuestionnaire} or remote-controlling \cite{Chattopadhyay:OfficeSocialRemoteControl}, no system known to us has combined as many interactive mechanisms in one application as ours and allowed seemless integration between them, which sets the present approach apart.

\section{Classroom related}

\begin{figure}
\centering
\includegraphics[width=.65\textwidth]{iclickers}
\caption{\emph{i>clicker} devices, used in \cite{Chamillard:StudentResponseSystem}. Image source \cite{iclicker}.}
\label{fig:related-work-iclicker}
\end{figure}

As growing class-sizes have caused student participation to sink drastically \cite{Bry:Backstage}, researchers have tried to deploy mechanisms to make lectures more interactive and engaging. The first ap\-proa\-ches in this field of student-response-systems (SRS) utilised so-called clickers (see figure \ref{fig:related-work-iclicker}) -- remote-control-like devices, connected to a receiver station via radio frequency technology \cite{cuclickers:faq} which can be used for tasks like taking attendance and voting \cite{Chamillard:StudentResponseSystem}. Using these clicker systems has shown to ``yield a strong and positive relationship with student learning'' \cite{Chamillard:StudentResponseSystem}. However, the limitations of clickers -- the need for proprietary hardware and the limited interface consisting only of a few buttons -- lead researchers to experiment with personal mobile devices as input instead. In 2007 Lindquist et al. \cite{Lindquist:ExploringMobilePhonesActiveLearning} presented a system integrated with the University of Washington's Classroom Presenter software, which lets students submit answers to assignments and in-class quizzes via SMS and MMS or using their laptops. Although the mobile phone users struggled with the input of longer messages, they perceived the ubiquity and concenience of using a light-weight personal device as an advantage. Most students, however, were worried about the costs of using SMS or MMS as a requirement in class -- a concern modern devices with internet access and cheap data plans dispel. The first of these web-based approaches were explored around the same time. Esponda \cite{Esponda:ElectronicVotingOnTheFly} for example describes a system in which iPods and other wifi-enabled devices can be used to answer questions during class. What is interesting about her approach is not only the technology used, but also that questions do not have to be prepared in advance, but can also be created on-the-fly, using a pen-based tablet, resulting in more lively and spontaneous student-teacher-interaction.
The creators behind \emph{i>clicker}\footnote{\url{http://www1.iclicker.com/}}, the clicker system used in \cite{Chamillard:StudentResponseSystem}, have also recognised the shortcommings of their hardware-approach and now build mobile apps for students' personal devices. Like\cite{Esponda:ElectronicVotingOnTheFly}, their application makes it possible for lecturers to prepare quizzes beforehands or create polls on-the-fly to monitoring the students' knowledge, understanding and progress. Although also available as iOS and Android app, like most modern approaches, the i>clicker software also has a web version, making use of modern browsers' possibilities and the device-independence of the web as a platform.
The tool \emph{ASQ} \cite{Triglianos:InteractiveWebPresentationsImpress} lets lecturers create HTML5 presentations with \emph{impress.js} \cite{impressjs} which are then distributed to listeners via a link. Students follow the presentations on their mobile devices, and can submit questions connected to the current slide to the speaker. Quizzes (both open questions and multiple-choice) can be embedded in the slides by the teacher. These quizzes can either be graded automatically (for coding assignments and multiple-choice questions), corrected by teaching assistants or by the students in self or peer-assessment. While this project has put a lot of effort into the server-side and administration of slidesets, the present work concentrates more on the client-side and does not provide slide management tools. In contrast to our implementation, however, this approach lacks 
Another interesting approach is presented by Cheng et al. \cite{Cheng:TreebasedOnlinePresentations}, who propose a system which generates HTML presentations from \emph{Microsoft PowerPoint} slides and lets viewers add their own content (either additional material or questions) as vertical sub-slides. This way a tree-like structure is created in which teachers and students collaborate in interactive presentations. This architecture also inspired the sub-slide based presentation space deployed in this software.

Another popular application, with richer audience-spea\-ker-in\-ter\-ac\-tion and an emphasis on listener-listener-interaction is \emph{Backstage} \cite{backstage}. As digital backchannels like Twitter can foster the sense of community within the audience, but are usually hard to follow for presenters, Bry et al.\cite{Bry:Backstage} developed a backchannel specifically for large classrooms. Students can post messages publicly and send private messages to their colleagues. These public posts can be up or down-voted, as well as marked as unrelated. Together with an ageing-algorithm, this community feedback is used to estimate a post's relevance. Important feedback is then presented to the lecturer, to allow him or her to get a better sense for the audiences' opinion and understanding. Additionally, small quizzes and polls serve as performance feedback to the teacher and students. Though one of the most mature systems studied for this thesis, having been developed specifically for classrooms, the use of the software in other scenarios is not ideal. Moreover, most of the features concentrate on listener-listener-interaction, while the present thesis focuses on mechanisms strengthening the speaker-audience-interaction.

Like Backstage, most of these approaches sound promising but are tightly bound to an educational context. The project discussed in this thesis, however aims for a broader field of application and concentrates on business-settings.

\section{Office environments}

In contrast to classroom-related software, meeting-en\-vi\-ron\-ments usually have an significantly lower amount of participants, as well as a smaller gap between the speaker and the audience. Another difference lies in the polling, surveying and quizzing functionality most of the presented projects offer: while these usually have only one correct answer in educational settings, to grade students \cite{Lindquist:ExploringMobilePhonesActiveLearning, Triglianos:InteractiveWebPresentationsImpress, Bry:Backstage}, the goal in meeting environments is to make decisions and collect ideas, without judgment and often anonymously.
The systems we want to quickly introduce all have a focus on mobile devices and their usage in meetings and office-related presentations and curiously were all developed by Microsoft Research: In \cite{Bohmer:SmartphoneUseRude}, as well as examining the perception of smartphone use in meetings, the mobile application \emph{Meetster} is presented. The study finds that although people primarily use their phones for meeting or work-related tasks, they tend to think their colleagues use theirs for private purposes. Unlike the present thesis, in which mobile devices should be used in the context of presentations, \emph{Meetster} was developed to help getting to know other meeting attendees in a playful way. This changed the perception of using one's smartphone during the meeting and was described as ``fostering social interactions''. While the findings of the study conducted as part of this publication legitimise our approach, this thesis presents a more practical approach, more relevant to the presentation itself, instead of just connecting meeting attendees through a game.

\begin{figure}
\centering\small
\begin{tabular}{cc}
\includegraphics[width=.5\textwidth]{feedback-meetings-ms-research}
 &
\includegraphics[width=.144\textwidth]{feedback-meetings-ms-research-app} \\
(a) & (b)
\end{tabular}
\caption{\emph{Crowd Feedback} \cite{Teevan:MobileFeedbackDuringPresentation} used during a presentation (a): The bar next to the slides shows one dot per participant in the meeting, the dots can be controlled with the app (b). The feedback dots fade out over time. Image source \cite{Teevan:MobileFeedbackDuringPresentation}.}
\label{fig:related-work-crowd-feedback}
\end{figure}

A system concentrating more on presentations directly is \emph{Crowd Feedback} \cite{Teevan:MobileFeedbackDuringPresentation}, a piece of software which allows listeners to give a speaker continuous, real-time feedback, using their personal devices. A responsive web application with a like and dislike button controls the feedback-system. The participants' reactions are shown with a red (dislike) or green (like) dot for each attendee in a sidebar next to the presentation slides (see figure \ref{fig:related-work-crowd-feedback}). An evaluation of the system showed that the participants felt more engaged with the presentation and connected to other listeners. Many users stated only having the possibility to like or dislike did not reflect enough options and that a button related to the speech pace might have helped. It was also noted that the sidebar was perceived as disturbing and made it harder to pay close attention to the presentation. This study and its conclusions have inspired the implementation of an instant feedback mechanism for listeners in the present work, however, instead of only having the binary like and dislike, the reactions are based on emojis, allowing for more insightful and faceted feedback.

The third study concerns itself with the navigation through slides: \emph{Office Social} \cite{Chattopadhyay:OfficeSocialRemoteControl}, a PowerPoint plugin with companion smartphone app, allows presenters and listeners to navigate through PowerPoint slides using their mobile phones. Members of the audience can either browse the slides privately, or take over the control of the presented slides, allowing them to effictively steer the presentation or discussion. As in the present approach, Chattopadhyay et al.'s software allows members of the audience to review the slides privately, making it possible for latecomers to catch up and to generally estimate the length and direction of the talk \cite{Chattopadhyay:OfficeSocialRemoteControl}. However, as their interface focuses on the navigation between slides, the preview of the slides is fairly small. Our approach tries to focus on the content of the slide and instead of offering big buttons to navigate around, makes use of intuitive swipe gestures, which can potentially be used eye-free more easily \cite{Negulescu:TapSwipeMove}. Another disadvantage is the overhead of having to download a smartphone application before the start of a presentation, as well as the limitation of the application only being available for Windows Phones.

\begin{figure}
\centering
\includegraphics[width=.3\textwidth]{office-social}
\caption{\emph{Office Social} \cite{Chattopadhyay:OfficeSocialRemoteControl}'s smartphone app interface. A preview of the slide is shown on top, followed by big buttons for navigating between slides. In the left picture, the application is in \emph{review} mode, where a local copy of the slides can be navigated through. By pressing a button in the interface the \emph{interaction} mode is activated, allowing listeners to navigate through the master slides (right). Image source \cite{Chattopadhyay:OfficeSocialRemoteControl}.}
\label{fig:related-work-crowd-feedback}
\end{figure}

\section{General presentations}
\label{sec:related-work-general}
Since lectures and meetings both are very specific forms of presentations, a few paragraphs should also be dedicated to general approaches in this third section.
One publication, which concentrates on polls and their real-time evaluation and rendering is \cite{Inoue:RealTimeQuestionnaire}: Inoue et al. present a system which distributes \emph{Microsoft PowerPoint} presentations using modern web-technologies while making it possible to alter and update the slides in presentation mode. This way questionnaires can be answered and their results displayed in real-time. Additionally, members of the audience can add annotations (both handwritten and digital) to slides. Although this approach seems very promising, pictures, videos and other types of media are ignored entirely. Moreover the interface seems too complicated for small devices and is therefore only usable on laptops and maybe tablets.

Two more products, though not subject to scientfic research and more commercial than the approaches presented so far, are \emph{Mentimeter} \cite{mentimeter} and \emph{sli.do} \cite{slido}. Both tools are web applications with real-time polling support, usable in any presentation. Both systems work very similarly: listeners go to the respective website and enter a presentation code to then be connected to the live voting. A handy feature Mentimeter offers is to query the device's location to determine the right presentation. Sli.do on the other hand also supports questions from the audience, which can be up-voted by the listeners, making it easy for speakers and participants in podium-discussions to answer the most relevant questions. Moreover, additionally to multiple-choice polls, sli.do also supports open questions and ratings. While the creators of Mentimeter provide a PowerPoint plugin, sli.do is not directly linked to any presentations. However, the popular canvas-based presentation-tool prezi \cite{prezi}, offers seamless integration with the application. It is worth noting that prezi itself already offers mobile features out of the box: Presentations can be controlled remotely from the speaker's phone or tablet as well as be viewed and followed by members of the audience in real time, using a mobile application.

More web-based presentation tools include \emph{Google Slides} \cite{google-slides} and \emph{PowerPoint Online}\footnote{\url{http://office.live.com/start/PowerPoint.aspx}}. While PowerPoint Online seems to only offer a simplyfied version of the desktop application online, Google Slides also provides mobile features such as editing and authoring slides on phones or tablets and controlling them remotely.

To conclude this chapter, a few words should also be said about the JavaScript presentation library \emph{reveal.js} \cite{revealjs} and its accompanying visual edior \emph{slides} \cite{slides-com}. Reveal.js offers features such as remote controlling slides for the speaker and following presentations on personal devices for members of the audience. However, the installation to achieve the latter so-called \emph{multiplexing} functionality, is fairly complex and involves setting up a socket-io server, running the master-presentation statically and locally and uploading a client version of the presentation to a publicly accessible server. Reveal.js offers a reliable online presentation library and could have served as a starting-point for the project presented in this thesis. However, due to their closed environment, tightly coupled code and lacking support for extensibility, we decided to instead implement an own presentation library (see chapter \ref{cha:implementation}, section \ref{sec:implementation-technologies-unveil}).
\chapter{Interactive Mechanisms}
\label{cha:mechanisms}

In a first step of identifying possible mechanisms which could make presentations more engaging and interactive, we analysed different types of presentations. There are several factors which determine these types, such as the size of the audience, the environment around and purpose of the presentation as well as the speaker and audience. In the following these factors will be described shortly to then present and discuss mechanisms these could profit from most.

\section{Factors}

\subsection{Audience Size}
One aspect which plays an important role in the type of presentation and thereby the interactive mechanisms applicable is the size of the audience. Different challenges present themselves, depending on the amount of listeners: While there might be a debate between speaker and audience in small group sizes, it is hard for audience members to directly communicate with a speaker during conferences or in large lecture halls. Shy or introvert attendees might remain unheard \cite{Bry:Backstage} and only a usually randomly chosen subset of people get the opportunity to ask audience questions after talks in conferences. At the same time estimating the audience's knowledge and interest as well as the general mood gets increasingly difficult both for the presenter and attendee as the number of participants rises. Additionally to the interaction between speaker and audience, another important factor is listener-listener interaction \cite{Moore:ThreeTypesOfInteraction}. Group-dynamics largly depend on the audience size and smaller groups usually perform better than big ones \cite{Phillips:GroupProblemSolving}.
The general conclusion therefore is that big audiences struggle to connect and interact with the speaker and each other and interactive tools must aim to strengthen the bidirectional bond between presenter and listeners. In smaller groups, on the other hand, the focus should be put on supporting the already existing dialogue and exchange between all participants of the presentation. As peer-pressure might rise in smaller groups and the better listeners know each other, ways of anonymously contributing to the outcome or flow of a presentation become more important.

\subsection{Presentation Environment}
% Remote vs co-located!
% setting: classroom vs. meeting vs. conference etc.
The environment of a presentation is described by all factors surrounding the presentation. One of them is the setting a talk is given in, in other words, if it is embedded in a meeting, a talk at a conference or a lecture at school or university. Other aspects worth considering are whether the audience is co-located or distributed and which technologies are available. As this work concerns itself only with mobile devices in the context of co-located presentations, difficulties added through remote presentations as well as missing technical equipment will be disregarded in this section. Instead, a closer look will be taken at the setting: In a lecture, it is desirable to measure the students' participation and engagement, as well as their understanding of a topic. In meetings, on the other hand, interactive mechanisms are more likely to aim for the promotion of collaboration between all participants. Conferences might search to foster the interactivity between attendees, to support networking. Instead of taking all possible scenarios into consideration, this work concentrates on business-related settings and explores mechanisms which foster collaboration.

\subsection{Presentation Purpose}
When speaking of the purpose of a presentation, McClain \cite{McClain:TypeOfPresentations} identifies four major types: informational, motivational, pursuasive and sales. According to him, informational presentations search to educate the listeners, while motivational speeches try to inspire the audience to take action. Pursuasive talks usually present new ideas or directions and have the goal of making the listeners re-think old approaches and consider or even embrace new ones. Sales presentations, lastly, often use elements of the other three categories with the aim of ``obtaining a decision at the presentation's end'' \cite{McClain:TypeOfPresentations}. While motivational, pursuasive and to some extend sales presentations often operate on an emotional level in the present moment, informational talks often include a way for listeners to re-visit the taught material through transcripts, lecture notes or handouts. Moreover, motivational, pursuasive and sales presentations focus on the goal of getting the audience to take action and therefore put more emphasise on the listeners than content-centric informational speeches. This creates two very distinctive needs for interactive mechanisms: on one side the ability for the audience to actively shape the path of the presentation, on the other hand the possibility to re-visit presentation slides (potentially including notes and additional material), after the end of a talk.

In the context of business-settings, as discussed in the present thesis, one should also take meeting-types into consideration. Böhmer et al. \cite{Bohmer:SmartphoneUseRude} differentiate between conversations, status updates, presentations (``scheduled session with higher level of formality'', not to be confused with the more general concept of presentations examined in this thesis), brainstorming and training. All these can be prepared beforehand and have different requirements and challenges to overcome: In brainstorming sessions, for example, it is vital for every team member to be able to contribute to the end result; in trainings, it is hardly possible to find a pace that fits all participants, etc.
% welches material wird mitgenommen? wie können lectures erinnert werden? welches material nehmen wir mit? --> nachbereitung/handouts!
% motivational und pursuasive: mehr im hier und jetzt, memorable auf gefühlsebene!
% in manchen ist es einfacher das Publikum leiten zu lassen als in anderen (motivational & pursuasive)

\subsection{Speaker and Audience}
% tech savvyness for audience
% spontanität and flexibility for speaker --> when do I take questions? überfordert? etc.
% level of formality!
The last factor taken into consideration in this chapter are the speaker and listeners themselves. Depending on the individual interest, but also character traits such as introversion, listeners will be more or less likely to engage in a presentation actively \cite{Bry:Backstage}. The inter-attendee relationship as well as the relationship between attendees and speaker also plays a role in which mechanisms are appreciated and which are not \cite{Moore:ThreeTypesOfInteraction}: while it is common for listeners to jump into the role of the presenter in meetings with flat-hierarchies, the same behaviour is a rare sight in lectures or might even be deemed inappropriate or rude in more formal settings. When taking the speaker into consideration, the set of tools needed are more sophisticated than the ones necessary to only follow a presentation: Foremostly, speakers need a way of navigating through slide decks. It is desirable to have an overview of the entire presentation and while listeners only concentrate on the current slide, many speakers rely on notes or use timers, which also need to be placed in the interface.
Moreover, the presenter's personal traits, experience and bluntly talent, play a central role in the successful deployment of interactive mechanisms: the flexibility, confidence and technological expertise of a presenter all determine how distracting or even stressful certain features are perceived as and whether a speaker is able to react to these spontaniously \cite{Wacker:PresenterExperience}. It is therefore crucial to give speakers the ability to turn said mechanisms on and off. An important challenge which also arises with this question is how to design these mechanisms in a way that is neither perceived as intrusive nor interrupting (this will be discussed in more detail in chapter \ref{cha:design}).
To summarise, when developing interaction tools, it is vital to take a participant's personality and their relationship to other ones into consideration. In the context of presentations, shy listeners should be given tools to make them heard; presenters need full control of the mechanisms provided.

\section{Resulting Mechanisms}
With these aspects and challenges in mind, a multitude of different mechanisms can be derived. Although many more are thinkable, this section concentrates on the ones implemented in course of the thesis project. However, we will try to point out other possible features and provide resources to projects focusing on these. One point to keep in mind is that not all of the presented mechanisms work equally well in every environment but instead have scenarios they are best suited for and others in which they are practically rendered redundant.

% Speaker: Remote control
% Speaker: See next slide right and down, notes on phone or tablet/laptop!
%
% Following slides on personal device
% Questions
% Anonymity
% Polls
% Reactions (Emojis) -- tell how these can engage the people more, not only about how they provide feedback to presenter!
% Annotating slides and sharing content (!) --> not really studied yet, so cool!

\subsection{Remote Control}
One mechanism of special importance for speakers is the ability to control slides and navigate through them. As many presentations involve more than just one speaker and can profit from sharing control over slides with others \cite{Chattopadhyay:OfficeSocialRemoteControl}, any amount of presenters should be able to be connected at any given point. Controlling should be possible from any personal device, may it be a laptop, tablet or mobile phone, giving the speakers maximal freedom.

While using arrow-keys in a desktop environment feels natural to navigate between slides, the native equivalent on touch-devices are swipe gestures. These are more accurately and faster when operating a phone with one hand \cite{Lai:SingleHandedThumbInteraction} and are less prone to error when used eye-free \cite{Negulescu:TapSwipeMove} and should therefore be prefered over buttons, clustering the interface.

\subsection{Following Slides}
% also gives latecomers the chance...
For members of the audience, one important feature, both in the context of re-visiting slides, to accommodate individual learning paces \cite{Cheng:TreebasedOnlinePresentations} and even to give latecomers a chance to catch up with the presentation \cite{Chattopadhyay:OfficeSocialRemoteControl}, is to be able to independently follow the slides. This should again be possible on any personal device and focus on the slide content, in a way that maintains the readibility of all text. The mechanism can be designed in many different ways and could even allow listeners to remote control the presentation \cite{Chattopadhyay:OfficeSocialRemoteControl}, our implementation however only provides individual slide navigation on the personal device. Additionally, the progress of the presentation should always be synchronised with the individual devices, allowing listeners to truely follow along. This basic mechanism can be extended to offer features such as turning the synchronisation on and off (effectively allowing to navigate freely and jump back to the presenter's state) or to only allow listeners to see the last slide the presenter has already shown.

\subsection{Paths}
Also connected to navigation and following slides is the possibility to offer different paths through the presentation. Especially in informational talks these can account for different backgrounds and levels of knowledge in the audience, they however, also make it possible to get listeners more involved in shaping the presentation. Paths should both be accessible to each audience member individually (for further reference or to catch up on a topic), as well as on the projector (e.g. by polling, as discussed in the next subsection).
The possibility to flexibly navigate through a presentation has proven to be one of the biggest advantages of canvas-based presentations \cite{Lichtschlag:CanvasPresentationsInTheWild} and have a wide field of application. The scenarios this thesis concentrates on are the following: On one hand, the paths can cover different levels of details (e.g. \emph{overview}, \emph{regular} and \emph{detailed}), as well as providing a way of skipping certain slides without having to navigate through all of them (e.g. skipping the introduction). Another option would be to let the audience decide between entirely different topics, depending on their personal interest. While canvas-based presentation tools like Prezi innately offer this flexibilty, slide-based tools often only make this behaviour possible by manually skipping over slides, which can interrupt the flow of the presentation \cite{Dieberger:NarrativeFlow}. PowerPoint extensions enabeling advanced forms of navigation, as well as the presenter looking through slides before projecting them are discussed in \cite{Dieberger:NarrativeFlow}, \cite{Nelson:PalettePaperInterface} and \cite{Signer:PaperPoint}, the latter, however, will not be part of our implementation.

\subsection{Asking Questions}
Another feature, well-suited for informative talks, is the possibility for members of the audience to ask questions. Another scenarios are big crowds, in which it is hard to be heard as an individual. A tool specifically designed for such settings is sli.do, which was already introduced in chapter \ref{cha:related-work} section \ref{sec:related-work-general}.
More generally, such mechanism should enable members of the audience to submit questions for the presenter to answer. These questions should either be displayed directly, or collected for the presenter to go through at the end of the presentation, depending on their preference and flexibility. This mechanism also highly depends on the presentation environment: In a classroom, questions should be answered immediately, while conferences usually only allow them at the end of talks. Questions could moreover only be visible to the presenter, or every participant. Concentrating on business-related settings, we propose a question feature which allows audience members to submit questions at any point of the presentation. These should be accessible for every attendee, to spark others' interest and participation. From the presenter's point of view, questions should be displayable instantly, at the end of the talk or any time inbetween, leaving the decision when to react to questions to each individual speaker.

% what does it mean for the presenter - take questions right away or in the end?

\subsection{Polls}
Another possibility to ask questions is polling. Although polls might also be generated by listeners, we propose a mechanism which lets the speaker create them. To give presenters more flexibility and because questions often only arise during talks \cite{Esponda:ElectronicVotingOnTheFly}, these surveys should be creatable in the preparation for a speech as well as on-the-fly, during presentations. This mechanism can help getting to know ones listeners (relationship between listeners and speaker), as well as estimate a crowd's mood (big audiences) and is especially useful in combination with paths. If supporting anonymous voting, relying on electronical aids instead of raising hands can also be benificial in smaller groups \cite{Esponda:ElectronicVotingOnTheFly}.
While a big number of different polling mechanisms are conceivable (open questions, ratings, multiple choice, as well as different ways of visualising the results), single choice voting and visualisation in a bar-chart serve as a starting point for our approach. Another detail lies in when the results are presented: they can either be rendered as soon as a user chooses his or her answer or only after everybody has given their votes.
To summarise, the identified requirements for such mechanism are creation beforehands and during the presentation, real-time polling and data-visualisation as well as anonymity of the voting process.

\subsection{Reactions}
As described before, especially bigger crowds suffer from a lack of interaction possibilities between speaker and audience but also between members of the audience. While the latter is discussed in \cite{Bry:Backstage}, the present work focuses on the interaction between speaker and listeners. Besides the difficulity of asking questions, which was already covered, the main problem for the presenter is to estimate the crowd's mood, which is why we suggest a mechanism that lets attendees send real-time feedback to the speaker. This functionality is based on \cite{Teevan:MobileFeedbackDuringPresentation}; as highlighted by Teevan et al., however, their simplistic approach of just offering \emph{likes} and \emph{dislikes} is not faceted enough to represet the full range of emotions listeners can feel during a presentation. It is therefore important to provide more detailed feedback.
These reactions can either be displayed only to the speaker, or to the entire audience. The latter might distract listeners \cite{Teevan:MobileFeedbackDuringPresentation}, however, also holds the potential to encourage others to also react to the current slide and strenghten listener-listener bonding. While this mechanism is expected to work well in bigger crowds, it will likely introduce an unnecessary technical burden to smaller groups, in which it is easier to estimate the attendees' mood. 

\subsection{Content Sharing}
In contrast to live reactions and questions, content sharing is especially suited for smaller audiences. As discussed before, tools for smaller groups should strengthen the already possible dialogue between all participants. These scenarios make it possible for listeners to actively get involved in the presentation and not only shape the path through, but also the content of such. While adding subslides to a slide deck after a presentation \cite{Cheng:TreebasedOnlinePresentations} and text-based annotations \cite{Inoue:RealTimeQuestionnaire, Myers:CollaborationPDAs} during talks have already been discussed in previous work, to our knowledge, no other study has concerned itself with the possibility of adding listener-generated slides and multi-media content in live presentations. While being an exciting opportunity to explore a widely untouched research subject, this mechanism empowers listeners and transforms presentations entirely by combining classic slides with brainstroming-like interactions and related multi-media content. While the potential of this mechanism will be further discussed in chapter \ref{cha:conclusion}, the requirements for this functionality should shortly be defined: It should be possible for any listener to add their own content to any slide. This content includes text, websites (per link), videos and uploaded images (e.g. taken with their personal devices). Presenters should have a way of deciding whether to accept the contribution and if it should be added as a subslide or main slide. Moreover, this mechanism requires a lot of flexibility from the speaker, which is why it is important to allow them to turn off or silence the functionality, providing sensible fallbacks. While content sharing can transform a presentation into an interactive and collaborative effort in smaller groups, the functionality will likely lead to chaos in big groups, without further interface changes.

Now that the implemented mechanisms are clarified and their requirements defined, the next chapter deals with the design and user experience of the application.
\chapter{Application Design}
\label{cha:design}

% Application Flow, Modes, How does the application work in general?
% Design aspects, sketches, wireframes, thoughts and überlegungen behind design details

% Wacker: research has shown that technological problems are the main reason for negative presentation experiences for the speaker. ``Therefore, presentation software should take special care to avoid technological problems and assist the presenter if they should occur''

% We believe it is still important for the speaker to have an overview of the current slide as well as upcoming ones and be able to see notes on the presenter interface

% General requirements: be viewable on every device, work in real-time!

% Emojis: Find more studies about this! think: facebook, the conference Paulo went to etc. (maybe add photo of audience showing emoji faces!)
\chapter{Implementation}
\label{cha:implementation}
% short overview; why web -- use references!
% speak about quick prototyping possibilities and how it's easier to roll out updates and how nobody needs to download an app on their phone to interact with the presentation

\section{Project Scope}
\label{sec:implementation-scope}
% What's the general scope of the project? Why is everything on the client and 
% not the server? etc.
As the aim of the present work is to explore ways of incorporating mobile devices into presentation workflows, the goal of the project was to use an easily extensible presentation library to then build the mechanisms discussed in the previous chapter \ref{cha:mechanisms}.
As the focus was placed on the interaction possibilities between speaker and audience, the creation of the presentation for the speaker or the management of slides and presentations were out of the project scope. Therefore the server used for connecting different users to the presentation was kept as simple as possible, allowing any potential other developer to work with their own servers and technology stacks.

In total, a system with several ways of interacting with the presentation from mobile or desktop devices was created, putting emphasise on mobile-optimised views and navigation possibilities. This system includes synchronisation of navigation state and state changes between viewers and speaker, the possibility to add sub-slides during the presentation for the audience, a speaker-view showing the next slides and controls, real-time voting (both created on-the-fly and prepared beforehands) and the possibility to create different paths through the presentation. In the following the technologies used in the project will be analysed and described to then go into details on the implementation, problems and solutions of the mentioned components.

\section{Technologies}
\label{sec:implementation-technologies}
% Which technologies were chosen and why?
% How do they generally work? To a level on which the reader can
% understand the rest of the implementation details
% A few words about responsive design and media queries would probs be good
% a few words about babel and es6

The project generally tries to follow best-practices in web development and utilises modern CSS3 and JavaScript features and frameworks. The software is written in ECMA\-Script\-2015, makes use of the \emph{node package manager}\footnote{\url{https://www.npmjs.com/}}(short \emph{npm}) for managing dependencies and \emph{Babel} to transpile to ECMA\-Script\-5. Additonally to relying on CSS3 features, this project also uses \emph{Sass}\footnote{\url{http://sass-lang.com/}} as a CSS pre-processor. Media-queries allow for mobile-friendly views.

On the front end, which this project focuses on, the JavaScript library \emph{React} is the framework of choice, additionally applying the \emph{reactive programming} paradigm using \emph{RxJS} to allow for a simpler interface for event-driven operations. The communication between client and server is handled by \emph{socket.io}\footnote{\url{http://socket.io/}}.
This section tries to introduce the reader to the main technologies used to establish a base on which the following technical implementation details can be understood.

\subsection[ECMAScript2015 and Babel]%
             {ECMAScript2015 and Babel%
             \protect\footnote{\url{https://babeljs.io/}}}%
\label{sec:implementation-architecture-es6}
% it's a recommendation, but it takes long until browsers implement it and users update their browsers.
JavaScript undoubtly is an integral part of front end web development and since the emergence of server-side JavaScript with Node.js and its package manager npm has developed into a programming language widely used by web developers. Both PYPL\footnote{\url{http://pypl.github.io/PYPL.html}} and TIOBE\footnote{\url{http://www.tiobe.com/tiobe_index}} programming language indices rank JavaScript among the top 10 programming languages (PYPL at 5, TIOBE at 7 at the time of writing). Stack Overflow's 2015 Developer Survey even places JavaScript as the number 1, most-used programming language with 54.4\% and JavaScript, Node.js and AngularJS all three rank amongst the top 5 languages developers expressed an interest in developing with (\cite{stackoverflow-developer-survey}).

However, like any front end technology, JavaScript suffers from slow end user adoption, as a multitude of browser versions exist for different devices and operating systems and many people still do not auto-update their browsers. Another factor is the time it takes for browser-vendors to implement new ECMAScript standards (the standard behind JavaScript) and roll out said updates. This is exactly what is happening with the new ECMAScript standard, ECMA-262, commonly known as ECMAScript 2015 or ES6: Although the General Assembly has adopted the new standard in June 2015 (\cite{ecma2015}), \emph{Kangax' ECMAScript compatibility tables}\footnote{\url{https://kangax.github.io/compat-table/es6/}} still show a fairly low level of adoption, especially among mobile browsers. ES6 makes JavaScript easier and more efficient to write by providing new semantics for default values, arrow-functions, template-literals, the spread operator or object destructuring (\cite{es6}). It also makes JavaScript easier to understand and safer to develop, with the introduction of block-scoped variables (\texttt{let} and \texttt{const}) and finally offers native support of modules and promises (\cite{es6}).
As these features are all included in the new ECMAScript standard, it is safe to assume browser-vendors will implement them in the near future. Until then, developers who want to already make use of them, can make use of so-called \emph{transpilers}, compiling ECMAScript 2015 code into ECMAScript 5, which is exactly what Babel does. With over $650000$ downloads in March 2015 (according to npm) and companies like facebook, netflix, mozilla, Yahoo or PayPall using it (\cite{babel-users}), Babel is the de facto standard solution transpile to ECMAScript 5 and was also chosen for this project.

\subsection[React]%
             {React%
             \protect\footnote{\url{https://facebook.github.io/react/index.html}}}%
\label{sec:implementation-architecture-react}
% explain base concept of having re-usable components and how they are defined!
% get some HTML code in there :)

\subsection{Reactive Programming}
\label{sec:implementation-architecture-rxjs}

\subsection[socket.io]%
             {socket.io%
             \protect\footnote{\url{http://socket.io/}}}%
\label{sec:implementation-architecture-socketio}
% problematic, talk about alternatives and problems in production, maybe even HTTP2
% e.g. safari conntection issues
% broadcasting functionality didn't quite work
% corporate firewalls can be a problem! big, in comparison to others

\section{Architecture}
\label{sec:implementation-architecture}
% A graphic explaining how the repos are built on top of each other would be great
% Which repos do I have and what do they include functionality-wise?
% Details on each repo!
% explain why different repos and how they are all own npm packages that can 
% easily be included in other projects

In general, the application is build on top of \textit{unveil.js}\footnote{\url{https://github.com/ostera/unveil.js}}, an open-source JavaScript library for presentations which was developed by Leandro Ostera and myself in the beginning of the project and which I extended and adapted to my needs during the project.

\subsection{Core library -- \texttt{unveil.js}}
\label{sec:implementation-architecture-core}
% explain how this was developed together with Leandro, explain general parts
% like router, navigator, UnveilApp
% and concepts like modes, presenters, controls
% give overview over controls in here.
% also talk about the mobile style sheets / responsive design and the TouchControls, which I also implemented alone.
% mention that this part is unit tested with jest(https://facebook.github.io/jest/)

\subsection{Network synchronisation library -- \texttt{unveil-network-sync}}
\label{sec:implementation-architecture-network-sync}

\subsection{Interactive library -- \texttt{unveil-interactive}}
\label{sec:implementation-architecture-interactive}

\subsection{Example application -- \texttt{unveil-client-server}}
\label{sec:implementation-architecture-client-server}
% how is everything defined? what has to be included?
% what are the steps of building an application with unveil?
% how are the slides defined? how are they styled? what about the modes?
% shortly talk about server and how any server could really be used for this.
\chapter{Results}
\label{cha:results}

This chapter offers insight into the outcome of our work. First, the final presentation system is presented, to then cover the results of our early user study. Since one of our objectives was to create a re-usable set of libraries, the usage of the unveil ecosystem from a developer's point of view will also be part of this chapter.

\section{Application}
\label{sec:results-user}
In course of this thesis, we have developed three designated interfaces, one for the listener, one for the speaker and one for the projector. While the projector interface does not have any specific requirements, both the listener and presenter view (see figure \ref{fig:results-user-speaker-presenter}) were optimised for devices with small screens.

Our system offers several ways of interacting with a presentation using ones personal, mobile devices:
\begin{itemize}
\item remote controlling slides,
\item following slides,
\item different paths through presentations,
\item voting and voting creation (on-the-fly),
\item instantaneous reactions and
\item multi-media content sharing.
\end{itemize}
%
With these mechanisms in place, presenters can navigate through slides without the need for proprietary hardware and capture the mood of the crowd through emoji. This is especially useful with increasing audience sizes, as it gets harder to address every participant. The same is true for votings, which can help the speaker to get an idea of the audience' knowledge and interest. When combined with different paths through the application, the progress of a talk or meeting can be controlled by the listeners, tailoring the experience to their needs. Due to our anonymous polling implementation, votings also offer an advantage over hand-raising in smaller audiences. In these settings, the content sharing mechanism shines: by inserting new main and sub-slides, listeners are given full control over the development of a presentation. Related content, notes, questions and comments can all be added by members of the audience, making presentations a collaborative effort. As all these features can be stressful for the speaker, all of them can be enabled or disabled, to their liking.
As the reaction feature of the application has already been covered in detail in chapter \ref{cha:design} and the other mechanisms are fairly self-explanatory, we will now take a detailed look at the voting and content sharing functionality.

\begin{figure}
\centering
\begin{tabular}{ccc}
\includegraphics[width=.165\textwidth]{presenter-mode-mobile} &
\includegraphics[width=.59\textwidth]{presenter-mode-desktop} \\
(a) & (b)
\end{tabular}
\caption{Speaker Presenter on mobile (a) and desktop (b).}
\label{fig:results-user-speaker-presenter}
\end{figure}

\begin{figure}
\centering
\includegraphics[width=.6\textwidth]{polls-modal}
\caption{Poll creation modal in presenter mode on tablets, laptops and desktops.}
\label{fig:results-user-polls-modal}
\end{figure}

\subsection{Voting}
Votings can be prepared beforehands or on-the-fly -- during the presentation in the speaker interface (see figure \ref{fig:results-user-polls-modal}(a)). After the creation of a poll, the voting starts with the presenter navigating their browser to it.
To ensure as many listeners as possible use their right to vote, all participants are locked to the voting screen for the time of the voting (i.e. until the speaker navigates away from the slide again) and cannot navigate individually anymore. Once a listener votes (see figure \ref{fig:results-user-polls-modal}(b)), they are presented with the current voting results, which are also displayed on the projector. Currently, only single-choice votings are supported by the system, however, free-text or mutliple-choice questionnairs as well as any other kind of form elements could easily be added in the future.

\subsection{Content Sharing}

\begin{figure}
\centering
\begin{tabular}{ccc}
\includegraphics[width=.168\textwidth]{sharing-listener-youtube} &
\includegraphics[width=.6\textwidth]{sharing-presenter-youtube} \\
(a) & (b)
\end{tabular}
\caption{Sharing of a youtube video. Request from listener on a phone (a) and accepting-modal in presenter-view on a computer (b).}
\label{fig:results-user-sharing-youtube}
\end{figure}

The seemingly biggest mechanism implemented in course of this thesis is the multi-media sharing feature. Three types of content can be shared in the current implementation:
\begin{itemize}
\item free text (comments and questions),
\item links (to websites, youtube videos or images) and
\item images from the filesystem.
\end{itemize}
The last also includes access to the camera and camera roll on smartphones (see figure \ref{fig:results-user-sharing-picture}) and through that offers a simple interface for uploading hand-written presentation notes, sketches and other related material.

The general process of sharing content is as follows: When a user presses one of the three content sharing buttons in the interface, a modal opens with a text area (free-text and link) or a file-upload (media). After filling in the necessary information, pasting a link or choosing a picture and confirming the action, the presenter instantely receives a content request (see figure \ref{fig:results-user-sharing-youtube}). If the speaker enables the \emph{do not disturb} mode, incoming requests will automatically be added as subslides to the page the request came from. Otherwise a modal will open in the speaker interface, displaying the content and offering to add the content as main slide, subslide or not at all. If the presenter chooses to accept the new content, a new slide will be added with a quote (free text), embedded video or website, or an image.

\begin{figure}
\centering
\begin{tabular}{ccc}
\includegraphics[width=.168\textwidth]{sharing-listener-1} &
\includegraphics[width=.168\textwidth]{sharing-listener-2} &
\includegraphics[width=.168\textwidth]{sharing-listener-3} \\
(a) & (b) & (c)
\end{tabular}
\caption{Sharing of a picture from an iPhone's browser (Google Chrome on iOS $9.3.4$) in listener mode. First, the listener opens the \emph{Media} sharing feature (a). Then a file can be chosen from different sources, such as the camera roll, or a new picture can be taken (b). In a last step, the picture is uploaded to the application and can then be shared (c).}
\label{fig:results-user-sharing-picture}
\end{figure}

In our opinion, all these mechanisms combined offer a good starting point for creating more interactive and engaging presentations. The potential of the developed features will be discussed in the next section.

\section{User Study}
\label{sec:results-study}

\begin{figure}
\centering
\includegraphics[width=\textwidth]{results}
\caption{User study results, per mechanism. Participants were asked to report their agreement or disagreement to given statements on a 5-point Likert scale.}
\label{fig:results-study-results}
\end{figure}

To measure the result of our work, we deployed the developed system in a presentation at the digital agency Oakwood Creative AB in Stockholm and asked the listeners to fill out a questionnaire (which can be found in the appendices). The results are summarised in figure \ref{fig:results-study-results}, a discussion follows in chapter \ref{cha:discussion}.

The information gathered from the questionnaire consisted of general information about the device(s) the participants used, statements regarding the different mechanisms, which were weighted on a 5-point Likert scale, as well as an area for general feedback for each feature. The given presentation had all interactive mechanisms activated and included a poll about how rude the listeners perceived smartphone use in meetings and presentations. There was, however, only one path through the presentation, so this mechanism was not covered in the user study. The entire questionnaire can be found in the appendices.
In total 9 people, all working in digital marketing, participated in the study. 7 of them used an iPhone to follow the presentation, one used a laptop. One participant interacted with the presentation both on a laptop and an iPhone. Most of the participants chose Safari as their browser on the phone (7 out of 8) and both laptop users chose Chrome.

The vast majority of users agreed or strongly agreed that the application was generally easy and intuitive to use (88.89\%). 77.78\% of the listeners also agreed or strongly agreed that the application felt fast and responded instantaneously. These results confirm the design of the user interface and our choice of technologies.

As far as following the slides on the mobile devices is concerned, the participants came to a similar conclusion: 75\% agreed (or strongly agreed) following the slides was intuitive, moreover 88.89\% reported they were able to see all necessary slide content on their devices. Furthermore, 77.78\% of the participants agreed or strongly agreed the mechanism made it possible for them to browse the slides at their own pace, which was the objective of this feature.

The polling feature received the most positive feedback: All users agreed (44.44\%) or strongly agreed (55.56\%) that the mechanism was intuitive to use, moreover participants reported feeling more connected to other listeners (55.55\%) and the speaker (77.78\% ). In total, almost 80\% of the listeners felt more engaged in the presentation through this mechanism (77.78\%), which confirms our approach.

The user acceptance of the instant reaction feature was also overwhelmingly positive: 87.5\% of the listeners (strongly) agreed the mechanism made it easier to give the presenter feedback. Moreover, we were initally concerned about the users' understanding of the emoji used. With 12.5\% of the participants agreeing and 87.5\% strongly agreeing they were familiar with the emoji and understood their meaning, this worry did not prove true.

Somewhat surprising results, however, were recorded with the content sharing feature. While 71.43\% understood the distinction between sharing links, text and images, only 57.14\% said they felt more engaged in the presentation and like they could actively shape the presentation, with 28.57\% respectively 42.86\% feeling indifferent. 85.71\% however, thought the mechanism was intuitive to use, which shows at least the user interface was well designed.

\section{From a Developer's Perspective}
\label{sec:results-developer}

Since one of our declared goals was the creation of reusable, extensible presentation libraries, a few words should also be spent on the usage of the system by other developers. This chapter will therefore go into details of how to set up an unveil presentation (section \ref{sec:usage-setup}), use the components (section \ref{sec:usage-components}) and lastly ways of customising, overwriting and extending behaviour (section \ref{sec:usage-customisation}). The code-examples of this chapter are based on the unveil-client-server repository \cite{unveil-client-server}.

\subsection{Setup}
\label{sec:usage-setup}
% how is everything defined? what has to be included?
% what are the steps of building an application with unveil?
% how are the slides defined? how are they styled? what about the modes?

Since the entire created code is available on npm, the first step in setting up an unveil presentation is to require the necessary libraries \texttt{unveil}, \texttt{unveil-network-sync} and \texttt{unveil-interactive}. In the entry point of the presentation (usually \texttt{index.html}), all bundled JavaScript and CSS-files are included and an HTML document is created which offers a tag that can be used to render the presentation (e.g. a \texttt{div} with the class \texttt{unveil}). For lower the page loading time \cite{yahoo-speeding-up-website}, script tags should generally placed in the \texttt{body} tag, usually before closing said tag.
As soon as this initial page is set up, the actual presentation can be built in an JavaScript file which should also be included here.
In this file, we will call it \texttt{index.js} from now on, all necessary libraries and components are imported:  \texttt{React}, \texttt{ReactDom}, as well as all unveil components that should be used. If any libraries which rely on the communication with the WebSocket server should be used in the presentation, the \texttt{SocketIO} singelton also has to be configured with the address of the server:
\begin{GenericCode}
import { createSocket } from 'unveil-network-sync'
createSocket('http://192.168.0.17:9000')
\end{GenericCode}

\subsection{Building a Presentation}
\label{sec:usage-components}

Once all libraries are imported, the actual presentation can be created. The most important component in this context is \texttt{UnveilApp}, which is imported from \texttt{unveil}. This component holds all the \texttt{Slide}s and is responsible for the configuration of the application (see section \ref{sec:usage-customisation}). Inside the \texttt{Slide} components, all content of the slide and the \texttt{Notes} are placed. Each of the slides will be rendered as common HTML and can include any number of other HTML tags and custom React components (see programm \ref{prog:usage-presentation-creation}). Although strictly-speaking not necessary, it is recommended to give slides (unique) names, since their name will be the id of the rendered HTML component and makes it possible to style components with CSS (see programm \ref{prog:usage-styling}). Additionally, if provided, unveil uses the name of the current slide as the url, allowing for text-based rather than index-based routes.

Other than that, slides can have a \texttt{left}, \texttt{right}, \texttt{up} and \texttt{down} property to allow for several paths through a presentation: All slides are provided as normal slide-sets, but the \texttt{left} and \texttt{right} attributes of the first and last slide of each path point to the previous and next slide shared by the entire presentation. \texttt{Link}s (available in the \texttt{unveil-interactive} package) can be used to access the first slide of each path. Additionally, the interactive extension also offers the components for preparing votings: \texttt{Voting}, \texttt{Question} and \texttt{Answer} (see programm \ref{prog:usage-voting}). The only necessary property for \texttt{Voting}s is the \texttt{name} attribute, which uniquely identifies the voting, as well as exactly one \texttt{Question}-child and an arbitrary number of \texttt{Answer}s.

\begin{program}
\caption{Example styling unveil slides using Sass. In this particular piece of code, the font family of all slides is set and a background image is added to the slide of name \texttt{start}.}
\label{prog:usage-styling}
\begin{CssCode}
.slide
  font-family: 'Open Sans'
#start
  background-image: url('../img/explore.jpg')
\end{CssCode}
\end{program}

\begin{program}
\caption{Creation of a presentation. Sets up two slides as an example. The DOM will be attached to the element of id \texttt{unveil} in the base HTML document.}
\label{prog:usage-presentation-creation}
\begin{JsCode}
ReactDOM.render((
  <UnveilApp modes={modes}>
    <Slide name="start">
      <h1>Unveil</h1>
      <h2>a meta presentation</h2>
    </Slide>
    <Slide name="intro">
      <img src="./img/problem.png" />
      <Notes>Explain initial situation</Notes>
    </Slide>
    ...
  </UnveilApp>
), document.getElementById('unveil'))
\end{JsCode}
\end{program}


\begin{program}
\caption{Creation of votings in unveil. The necessary components have to be imported from \texttt{unveil-interactive}.}
\label{prog:usage-voting}
\begin{JsCode}
<Voting name="like">
  <Question>Do you like these slides?</Question>
  <Answer value="yes">Yes</Answer>
  <Answer value="no">No</Answer>
</Voting>
\end{JsCode}
\end{program}

\subsection{Customisation and Extension}
\label{sec:usage-customisation}

\begin{program}
\caption{Mode definition for setting up an unveil.js presentation. Default (i.e. listener) and speaker modes are omitted to keep the example short, but generally follow the same pattern as the projector mode.}
\label{fig:usage-modes}
\begin{JsCode}
const modes = {
  default: {...},
  speaker: {...},
  projector: {
    controls : [
      NavigationReceiver, MediaReceiver, ReactionReceiver,
      VotingNavigatableSetter, VotingReceiver
    ],
    presenter: Presenter
  }
};
\end{JsCode}
\end{program}

Thanks to unveil's base architecture, it is possible for speakers to entirely customise the entire presentation logic. The most important step is to define the available modes and the presenter and controls which should be loaded in each of them (see figure \ref{fig:usage-modes}). Additionally to the existing controls in unveil and its current extensions, new (presentation) logic can be added by defining ones own React components and assigning them to modes. To interact with the WebSocket server, \texttt{SocketIO} is available from the network synchronisation layer. Data such as the current router state or slide, and the navigator or state subject are accessible through \texttt{UnveilApp}'s context. For examples of existing controls and presenters, the reader is adviced to refer to the implementation of the components discussed in chapter \ref{cha:implementation} \cite{unveil-fork, unveil-network-sync, unveil-interactive}.

Moreover, the \texttt{Router} and \texttt{Navigator} can also be entirely replaced by providing ones own functionality as \texttt{router} and \texttt{navigator} properties in \texttt{UnveilApp}. They only have to follow the same interface as the default implementations. If any additional configuration should be necessary (such as with setting the address of the WebSocket server or customising emoji), singletons or static methods of the components can be used.

\chapter{Discussion}
\label{cha:discussion}

A system such as the one developed in course of this thesis can be evaluated in many ways: The performance of the software can be quantified by measuring response times of the application, as happened in \cite{Niwa:Web-presentation-powerpoint} or \cite{Inoue:RealTimeQuestionnaire}, the usability can be assessed quantitatively through usability tests, measuring error rates and time to complete certain tasks and qualitatively using \emph{loud thinking} and interviews \cite{Reindl:automatisierte-user-interface-evaluierung}. If the expectations lied out in chapter \ref{cha:mechanisms} were actually met, if listeners feel more engaged and whether the perception of phone usage in presentations has changed with unveil can also be estimated using qualitative and quantitative evaluation methods. We feel, however, that valid results can only be achieved observing the usage of the presented tool over a longer period of time and involving control groups. Since this would go beyond the scope of this master thesis, we decided to refrain from a formal study and instead rely on the informal observation and evaluation of the system, both during and after working on the project.
Throughout the process of developing the different libraries, several informal presentations and meetings have been held using unveil, both in academic and business settings. This allowed for several iterations of the design, as well as the gradual introduction of more and more interactive mechanisms. Our findings and the strengths and the shortcomings of the project will be discussed in this chapter, followed by an outlook on future work.

\section{Usability}
\label{sec:discussion-usability}

We were pleased to see that all listeners have so far understood the base interface of the application, both on their phones and on the laptop. We have however not had a single tablet user in any of the presentations yet and most of the mobile phones were iPhones running Chrome or Safari. Some users experienced glitches in the navigation of the slides in Safari and could scroll within them, moreover local storage is disabled in Safari's \emph{private mode} (thus making it impossible to persist voting results). On bright side, initial concerns regarding the usage of socket.io have not come true, since the presentations were always held in networks we had full control over (so corporate firewalls did not play a role).

As far as the response time of the application is concerned, with up to 30 concurrent users, no noticable declines in performance have been experienced so far and the real-time features feel instantaneous. The only current limitation to this is the transmission of base64-encoded media content through the WebSocket. Since the delay is between listener and speaker, however, the transmission time is not critical for the application to feel responsive. Another point connected to sharing media is the absence of a native sharing feature for phones, which was an anticipated limitation of creating a web application over a mobile one.

The implemented functionalities were generally received well by the users, but utilised a lot more during meetings than informational presentations. The sharing mechanisms in particular seem to have caused most excitement about unveil. In our trials, however, we experienced a large amount of test-data being sent through the mechansisms, especially in the beginning of the presentation. We therefore advice others to provide one or two empty slides in the beginning of the presentation if the audience is new to the software, so the application and its functionality can be explored. Since some users mixed textual content with links in the early tests, we separated media upload, link sharing and questions (or other text) into distinct buttons. We were very pleased to see that some of the people who had used unveil in meetings before, later actively mentioned wanting to have the ability to share links and thoughts in meetings without the technology. Resulting from this feedback, an unexpected way of using the software emerged: Instead of enriching an existing presentation, an empty slide was provided, allowing anyone in a meeting to add their own content in the $2$-dimensional space in a brainstorming-like fashion.

The reaction-system was the one implemented last, which is why we are still missing more thorough insight into how listeners use it. In informal feedback rounds about the mechanism, the usage of emoji seemed to be understood well, however, these conversations were held with digital professionals and it will require further analysis to see the acceptence beyond other users. Another observation we made, is that although listeners generally reacted exceedingly positively to QR-codes (linking to the address of the presentation), it was usually faster for them to type the address into their browsers directly. This usually caused a few minutes of interruption, which should be accounted for.

The voting mechanism was also understood by users and no questions have yet arisen from it. To our surprise this functionality, like the possibility of having different paths through the presentation, did not seem to impress or excite users too much. This on one hand shows the implicitness of forms and voting mechanisms on the internet, on the other hand might indicate a potential for improvement.

The biggest weakness of unveil for most users was the inability for the presentation to be permanently altered. Especially in meetings and informational presentations, many listeners asked for a link to the collaboratively created presentation afterwards, some users were forced to reload the website due to cross-browser compatibility issues and lost the current state of the slides, late-comers also did not have the possibility to jump into an already altered presentation. This will make it necessary to create a more intelligent and opinionated back end in the future.

\section{Creation of Presentation}
\label{sec:discussion-dev}

Most of the feedback we have collected about unveil over the last months came from listeners. This has two reasons: Firstly, the creation of presentations requires knowledge and experience with front end web development technologies, secondly, we have not started promoting the resulting libraries yet, as we feel the system is not stable and mature enough to be used outside our internal settings. However, external developers have provided us with their feedback regarding the syntax used for defining slide decks and seem to not have had any problems understanding the usage of the libraries. This validates our decision to choose semantically-named tags rather than HTML class names to identify different components. On the downside, we seem to have overestimated the level of knowledge necessary to create own presentations, as a few developers were not entirely familiar with the new ES6 syntax and the process of bundling JavaScript and CSS files, so an easier way of importing all necessary libraries should be provided in an example presentation.
In the long run, this product will only be able to increase its popularity if a visual editor for authoring slides, as well as a system to manage (i.e. host) them will be available. Some users also raised the question of how and if it was possible to import PowerPoint presentations, to be able to use the created mechanisms in connection with already existing software.

\section{Architecture}
\label{sec:discussion-architecture}

Generally, it has to be said that the project is still only a prototype and does not provide the stability necessary for us to feel confident promoting the product. One particular problem we have been experiencing seemingly randomly is the navigation getting stuck in an infinite redirect loop when two or more connected presenters try to navigate at the same time. Moreover we have so far ignored security concerns such as adding password protection to the presenter mode.
Although the chosen architecture allows for a lot of flexibility and freedom for developers and in that sense fulfils the criteria impress.js and reveal.js did not, we have discovered more powerful, streamlined and widespread patterns when working with React over the last half year. Since we had no experience with React prior to the start of the development, best practices oftentimes only became apparent throughout the project and through the work on and with other React applications. To make the libraries yet easier to use and extend and more easily debuggable, future iterations of unveil will likely be based on \emph{Redux} \cite{redux}, with the presentation state globally accessible. This will make it easier to decouple components from each other through the introduction of a centralised state container.

\section{Future Work}
\label{sec:discussion-future-work}

The biggest outstanding improvement is definitely the persistence of updates to the presentation state for later revision by both listeners and presenter. For the presenter to profit even more from the new possibilities of interactive presentations, we would like to provide some analytics: When did the audience interact with the presentation most? When exactly were which reactions triggered? Which slides provoked the most input or questions and how long did audience members stay on each slide when browsing through them individually?

As far as the interactive mechanisms implemented in this project are concerned, initial tests have shown potential for improvement as well: the question sharing tool was used very seldomly, but instead the wish for a commenting or annotating functionality was mentioned by some users. Votings could offer several types (e.g. multiple choice, open questions, ratings), as well as different ways of visualising the results (e.g. pie charts, stars, clusters of answers). They could moreover be more tightly linked to paths through the presentation, so the result of a voting could directly link to a path.
Reactions generally need more testing and although we are content with the current functionality for the scope of this thesis, analytics for the presenter and time-based rather than slide-based display of reactions are desireable.
Overall, more animations could make the presentation feel even more responsive and offer a more \emph{native} feel for mobile users \cite{GoogleMaterialDesignGuide}.

From an architectural point of view, it is our declared objective to deliver a stable first version of the unveil ecosystem within the coming months. Redux and even further separation of concerns will make it easier to automatically test the application and add new features, such as the persistence of presentation state to local storage or a database. Moreover, porting the existing system to React Native and adding native sharing possibilities is worth a consideration.

A possibility not at all touched in the present work is the development of an authoring and hosting tools for unveil presentations. This would make it easier to create slide decks and render the necessity for front end development knowledge obsolete; effectively giving anyone a tool to create interactive presentations. This would also be possible offering imports of PowerPoint presentations or even re-building the interactive mechanisms as PowerPoint plugins.

Another opportunity which arose from the users' desire to share to a blank presentation during meetings as well as questions about annotations, would be to follow a canvas-based rather than a slide-based approach in future projects. That way meeting participants could effectively use the platform as a tool of collaboratively creating and sharing content, to brainstorm and take notes from any device in a shared working space.

Finally, more formal observations and long-term studies will be necessary to quantify the success of the developed mechanisms. Currently, the tool is mostly in use for informal Monday morning presentations with other digital professionals; the acceptance and usability of the application will have to be re-visited and re-evaluated when assessed with less technologically versed users.
\chapter{Conclusions and Future Work}
\label{cha:conclusion}

In this thesis, we present \emph{unveil}: an extensible JavaScript presentation eco\-system with a multitude of interactive mechanisms, aiming to make presentations more engaging, memorable and collaborative. At its core stand four React libraries, connecting presenters, listeners and projectors through a WebSocket server, making it possible for both audience and speakers to alter $2$-dimensional slide-sets in real time.
Different types of presentations were analysed to find their weaknesses and establish ways of enhancing the presentation experience: To make it easier to esitmate the listeners' level of knowledge, a real-time voting mechanism was implemented, allowing for both prepared polls and ones created on-the-fly during the presentation. Furthermore, the audience can instantaneously react to slides via emoji, giving the speaker an impression of the current mood. To account for individual learning-pace and late-comers, members of the audience can moreover browse and follow the slides on their personal devices as well as send questions to the presenter.
These questions, along with other multi-media content, such as pictures and links to websites and youtube videos, can then be embedded into the presentation as new main or sub-slides, effectively enabeling the audience to truely shape he presentation using nothing but the mobile devices they carry on them. This holds the potential of engaging listeners in the presentation, steering it towards certain topics, connecting members of the audience with each other and the speaker and adding related content for further reference.

Despite making presentations more memorable for the audience, this approach poses new challenges and requires more flexibility from the speaker. For this reason, all realised mechanisms can easily be activated an deactivated by the creator of the presentation. Since the created libraries were generally designed for other developers to re-use, modify and extend, we moreover offer ways of tailoring and configuring every last detail, from the routing logic over the emoji displayed to the mechanisms enabled for each user group (presenter, listener, projector).

\section{Future Work}
\label{sec:discussion-future-work}

Although long-term studies will be necessary to fully evaluate our approach and verify and quantify their success, the early user study of the system showed promising results. The users understood the interface and were especially excited about the voting mechanism. The initial evaluation of our prototype also showed room for improvement, amongst others, the persistence of the created presentation as well as the missing graphical user interface to create presentations. In future iterations of the libraries, the stability and testability of the platform will be improved as well as offering a way of permanently altering the presentation.

For the presenter to profit even more from the new possibilities of interactive presentations, we would like to provide some analytics: When did the audience interact with the presentation most? When exactly were which reactions triggered? Which slides provoked the most input or questions and how long did audience members stay on each slide when browsing through them individually?

As far as the interactive mechanisms implemented in this project are concerned, initial study has shown potential for improvement as well: the content sharing tool was used seldomly and the wish for a commenting or annotating functionality was mentioned by some users. Votings could offer several types (e.g. multiple choice, open questions, ratings), as well as different ways of visualising the results (e.g. pie charts, stars, clusters of answers). They could moreover be more tightly linked to paths through the presentation, so the result of a voting could directly link to a path.
Reactions generally need more testing and although we are content with the current functionality for the scope of this thesis, analytics for the presenter and time-based rather than slide-based display of reactions are desireable.
Overall, more animations could make the presentation feel even more responsive and offer a more \emph{native} feel for mobile users \cite{GoogleMaterialDesignGuide}.

From an architectural point of view, it is our declared objective to deliver a stable first version of the unveil ecosystem within the coming months. Redux and even further separation of concerns will make it easier to automatically test the application and add new features, such as the persistence of presentation state to local storage or a database. Moreover, porting the existing system to React Native and adding native sharing possibilities is worth a consideration.
While one of the presentations held with unveil so far has been remote, the possibilities of using this project with distributed audiences has yet to be studied. However, the opportunity of reacting to slides as well as adding content (potentially after a talk or meeting) are a promising foundation for remote presentations \cite{Isaacs:InteractivePresentationsDistributedAudience, Cheng:TreebasedOnlinePresentations} and should be examined further.

A possibility not at all touched in the present work is the development of an authoring and hosting tools for unveil presentations. This would make it easier to create slide decks and render the necessity for front end development knowledge obsolete; effectively giving anyone a tool to create interactive presentations. This would also be possible offering imports of PowerPoint presentations or even re-building the interactive mechanisms as PowerPoint plugins.

Another opportunity which arose from the users' desire to share to a blank presentation during meetings as well as questions about annotations, would be to follow a canvas-based rather than a slide-based approach in future projects. That way meeting participants could effectively use the platform as a tool of collaboratively creating and sharing content, to brainstorm and take notes from any device in a shared working space.

Finally, more formal observations and long-term studies will be necessary to quantify the success of the developed mechanisms. Currently, the tool is mostly in use for informal Monday morning presentations with other digital professionals; the acceptance and usability of the application will have to be re-visited and re-evaluated when assessed with less technologically versed users.

% Hier könnte man erwähnen dass das sharen mittels handy übers menü doch irgendwie besser wäre, aber den Nachteil hat dass man sich die app herunterladen muss.
% erwähnen: Könnte evtl. auch für remote presentationen verwendet werden!
% erwähnen: evtl. andere typen von Charts (Pie etc.) wie erwähnt in Electronic voting on the fly
% auszerdem: von power point exportieren wäre riesen Vorteil für Author!
% Wie wärs mit: Hinweis darauf, dass es wichtig ist die richtigen Mechanismen für die richtige Präsentation zu wählen um death by technology zu vermeiden (referenz zu \cite{Wacker:PresenterExperience}, dass failing technology reason #1 ist sich unwohl zu fühlen!). Mechanismen sollen nicht menschliche Kommunikation ersetzen sonder unterstützen und sind nicht immer nötig und mit Vorsicht zu genieszen
% Application for workshop environment (wie am Anfang besprochen! Großgruppe => Kleingruppe => Großgruppe)
% Sicherheitsconcerns nicht beachtet 
% Mehr tests benötigt, aber erste tests zeigen: Anfangsslide zum Austoben einbauen, App wird gut angenommen auch wenn Funktionen in Meetings mehr genutzt werden als in anderen Präsentationen
% Speichern der Resultate auf Implementierungsseite noch benötigt

%%%----------------------------------------------------------
%%%Anhang
\appendix
\chapter{CD-ROM/DVD  Contents}
\label{app:cdrom}

\paragraph{Format:}
		CD-ROM, Single Layer, ISO9660-Format%


\section{General}

\begin{FileList}{/}
\fitem{_DaBa.pdf} Master Thesis
\end{FileList}

\section{Copies of Online References}

\begin{FileList}{/references/}
\fitem{about-iclicker.pdf} About i>clicker \cite{iclicker}.
\fitem{babel-users.pdf} See who's using Babel \cite{babel-users}.
\fitem{bing-emoji.pdf} Do You Speak Emoji? Bing Does \cite{Bing:Emoji}.
\fitem{creative-bloq-hover.pdf} Hover is dead, long live hover \cite{hover}.
\fitem{cu-clickers-faq.pdf} CUClickers / i>clickers - FAQ \cite{cuclickers:faq}.
\fitem{ecma-262.pdf} ECMAScript 2015 Language Specification \cite{ecma2015}.
\fitem{facebook-reactions.pdf} Reactions Now Available Globally \cite{Facebook:Reactions}.
\fitem{github-reactions.pdf} Add Reactions to Pull Requests, Issues, and Comments \cite{Github:Reactions}.
\fitem{google-emoji.pdf} Google now also allows you to search using emoji characters \cite{Google:Emoji}.
\fitem{instagram-emoji.pdf} Emojineering Part 1: Machine Learning for Emoji Trends \cite{Instagramm:Emoji}.
\fitem{jsx-specification.pdf} Draft: JSX Specification \cite{jsx}.
\fitem{mozilla-file-api.pdf} Using files from web applications \cite{file-api}.
\fitem{npm-babel.pdf} npm's babel package page \cite{npm-babel}.
\fitem{prezi-presentations.pdf} The Science of Effective Presentations \cite{prezi-science}.
\fitem{react-benchmarks.pdf} More Benchmarks: Virtual DOM vs Angular 1 \& 2 vs Mithril.js vs cito.js vs The Rest \cite{react-benchmarks}.
\fitem{sm-es6.pdf} ECMAScript 6 (ES6): What's New In The Next Version Of JavaScript \cite{es6}.
\fitem{so-developer-survey.pdf} 2015 Developer Survey \cite{stackoverflow-developer-survey}.
\fitem{socketio-problems.pdf} Why We Ditched Socket.IO \cite{socketio-problems}.
\fitem{socketio-wildcards.pdf} Answer to Stack Overflow Question ``Socket.io Client: respond to all events with one handler?'' \cite{socket-io-wildcards}.
\fitem{yahoo-performance.pdf} Best Practices for Speeding Up Your Web Site \cite{yahoo-speeding-up-website}.
\end{FileList}

\section{Graphics}

\begin{FileList}{/graphics/}
\fitem{screenshots/*} Screenshots of unveil %
\fitem{wireframes/*.pdf}  Annotated Wireframes%
\fitem{wireframes/*.png} Original Balsamiq Wireframes%
\fitem{*.jpg, *.png} Original Raster Graphics %
\fitem{*.pdf} Original Vector Graphics %
\fitem{*.sketch} Original Sketch Files %
\end{FileList}


\includepdf[pages=-]{questionnaire.pdf}

%%%----------------------------------------------------------
\MakeBibliography
%%%----------------------------------------------------------

%%%Messbox zur Druckkontrolle
\include{messbox}

\end{document}
